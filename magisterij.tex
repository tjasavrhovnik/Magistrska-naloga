% !TeX spellcheck = sl_SI
% vim: set spell spelllang=sl:
% za preverjanje črkovanja, če se uporablja Texstudio ali vim
\documentclass[12pt,a4paper,twoside]{article}
\usepackage[utf8]{inputenc}  % pravilno razpoznavanje unicode znakov

% NASLEDNJE UKAZE USTREZNO POPRAVI
\newcommand{\program}{Matematika} % ime studijskega programa
\newcommand{\imeavtorja}{Tjaša Vrhovnik} % ime avtorja
\newcommand{\imementorja}{prof.~dr.~Franc Forstnerič} % akademski naziv in ime mentorja, uporabi poln naziv, prof.~dr.~, doc.~dr., ali izr.~prof.~dr.
\newcommand{\imesomentorja}{} % akademski naziv in ime somentorja, če ga imate
\newcommand{\naslovdela}{Minimalne ploskve}
\newcommand{\letnica}{2021} % letnica magistriranja
\newcommand{\opis}{}  % Opis dela v eni povedi. Ne sme vsebovati matematičnih simbolov v $ $.
\newcommand{\kljucnebesede}{} % ključne besede, ločene z \sep, da se PDF metapodatki prav procesirajo
\newcommand{\keywords}{} % ključne besede v angleščini
\newcommand{\organization}{Univerza v Ljubljani, Fakulteta za matematiko in fiziko} % fakulteta
\newcommand{\literatura}{literatura}  % pot do datoteke z literaturo (brez .bib končnice)
\newcommand{\sep}{, }  % separator med ključnimi besedami v besedilu
% KONEC PODATKOV

\usepackage{bibentry}         % za navajanje literature v programu dela s celim imenom
\nobibliography{\literatura}
\newcommand{\plancite}[1]{\item[\cite{#1}] \bibentry{#1}} % citiranje v programu dela

\usepackage{filecontents}  % za pisanje datoteke s PDF metapodatki
\usepackage{silence} \WarningFilter{latex}{Overwriting file}  % odstrani annoying warning o obstoju datoteke
% datoteka s PDF metapodatki, zgenerira se kot magisterij.xmpdata
\begin{filecontents*}{\jobname.xmpdata}
  \Title{\naslovdela}
  \Author{\imeavtorja}
  \Keywords{\kljucnebesede}
  \Subject{matematika}
  \Org{\organization}
\end{filecontents*}

\usepackage[a-1b]{pdfx}  % zgenerira PDF v tem PDF/A-1b formatu, kot zahteva knjižnica
\hypersetup{bookmarksopen, bookmarksdepth=3, colorlinks=true,
  linkcolor=black, anchorcolor=black, citecolor=black, filecolor=black,
  menucolor=black, runcolor=black, urlcolor=black, pdfencoding=auto,
  breaklinks=true, psdextra}

\usepackage[slovene]{babel}  % slovenščina
\usepackage[T1]{fontenc}     % naprednejše kodiranje fonta
\usepackage{amsmath,amssymb,amsfonts,amsthm} % matematični paketi
\usepackage{graphicx}     % za slike
\usepackage{emptypage}    % prazne strani so neoštevilčene, ampak so štete
\usepackage{units}        % fizikalne enote kot \unit[12]{kg} s polovico nedeljivega presledka, glej primer v kodi
\usepackage{makeidx}      % za stvarno kazalo, lahko zakomentiraš, če ne rabiš
\makeindex                % za stvarno kazalo, lahko zakomentiraš, če ne rabiš
% oblika strani
\usepackage[
  top=3cm,
  bottom=3cm,
  inner=3.5cm,      % margini za dvostransko tiskanje
  outer=2.5cm,
  footskip=40pt     % pozicija številke strani
]{geometry}

% VEČ ZANIMIVIH PAKETOV
% \usepackage{array}      % več možnosti za tabele
% \usepackage[list=true,listformat=simple]{subcaption}  % več kot ena slika na figure, omogoči slika 1a, slika 1b
% \usepackage[all]{xy}    % diagrami
% \usepackage{doi}        % za clickable DOI entrye v bibliografiji
% \usepackage{enumerate}     % več možnosti za sezname

% Za barvanje source kode
% \usepackage{minted}
% \renewcommand\listingscaption{Program}

% Za pisanje psevdokode
% \usepackage{algpseudocode}  % za psevdokodo
% \usepackage{algorithm}
% \floatname{algorithm}{Algoritem}
% \renewcommand{\listalgorithmname}{Kazalo algoritmov}

% DRUGI TVOJI PAKETI:
% tukaj

\setlength{\overfullrule}{50pt} % označi predlogo vrstico
\pagestyle{plain}               % samo številka strani na dnu, nobene glave / noge

% ukazi za matematična okolja
\theoremstyle{definition} % tekst napisan pokončno
\newtheorem{definicija}{Definicija}[section]
\newtheorem{primer}[definicija]{Primer}
\newtheorem{opomba}[definicija]{Opomba}
\newtheorem{aksiom}{Aksiom}

\theoremstyle{plain} % tekst napisan poševno
\newtheorem{lema}[definicija]{Lema}
\newtheorem{izrek}[definicija]{Izrek}
\newtheorem{trditev}[definicija]{Trditev}
\newtheorem{posledica}[definicija]{Posledica}

\numberwithin{equation}{section}  % števec za enačbe zgleda kot (2.7) in se resetira v vsakem poglavju

% Matematični ukazi
\newcommand{\R}{\mathbb R}
\newcommand{\N}{\mathbb N}
\newcommand{\Z}{\mathbb Z}
\renewcommand{\C}{\mathbb C}
\newcommand{\Q}{\mathbb Q}

% \DeclareMathOperator{\tr}{tr}  % morda potrebuješ operator za sled ali kaj drugega?

% bold matematika znotraj \textbf{ }, tudi v naslovih, kot \omega spodaj
\makeatletter \g@addto@macro\bfseries{\boldmath} \makeatother

% Poimenuj kazalo slik kot ``Kazalo slik'' in ne ``Slike''
\addto\captionsslovene{
  \renewcommand{\listfigurename}{Kazalo slik}%
}

% če želiš, da se poglavja začnejo na lihih straneh zgoraj
% \let\oldsection\section
% \def\section{\cleardoublepage\oldsection}

%%%%%%%%%%%%%%%%%%%%%%%%%%%%%%%%%%%%%%%%%%
%%%%%%           DOCUMENT           %%%%%%
%%%%%%%%%%%%%%%%%%%%%%%%%%%%%%%%%%%%%%%%%%

\begin{document}

\pagenumbering{roman} % začnemo z rimskimi številkami
\thispagestyle{empty} % ampak na prvi strani ni številke

\noindent{\large
UNIVERZA V LJUBLJANI\\[1mm]
FAKULTETA ZA MATEMATIKO IN FIZIKO\\[5mm]
\program\ -- 2.~stopnja}
% ustrezno dopolni za IŠRM
\vfill

\begin{center}
  \large
  \imeavtorja\\[3mm]
  \Large
  \textbf{\MakeUppercase{\naslovdela}}\\[10mm]
  \large
  Magistrsko delo \\[1cm]
  Mentor: \imementorja \\[2mm] % ustrezno popravi spol
%   Somentor: \imesomentorja   % dodaj, če potrebno
\end{center}
\vfill

\noindent{\large Ljubljana, \letnica}

\cleardoublepage

%% sem pride IZJAVA O AVTORSTVU  -- SE NATISNE V VIS

% zahvala
\pdfbookmark[1]{Zahvala}{zahvala} %
\section*{Zahvala}
% end zahvala -- izbriši vse med zahvala in end zahvala, če je ne rabiš

\cleardoublepage

\pdfbookmark[1]{\contentsname}{kazalo-vsebine}
\tableofcontents

% list of figures
% \cleardoublepage
% \pdfbookmark[1]{\listfigurename}{kazalo-slik}
% \listoffigures
% end list of figures

\cleardoublepage

\section*{Program dela}
\addcontentsline{toc}{section}{Program dela} % dodajmo v kazalo

\section*{Osnovna literatura}
Literatura mora biti tukaj posebej samostojno navedena (po pomembnosti) in ne
le citirana. V tem razdelku literature ne oštevilčimo po svoje, ampak uporabljamo
okolje itemize in ukaz plancite, saj je celotna literatura oštevilčena na koncu.
\begin{itemize}
  \plancite{lebedev2009introduction}
  \plancite{gurtin1982introduction}
  \plancite{zienkiewicz2000finite}
  \plancite{STtemplate}
\end{itemize}

\vspace{2cm}
\hspace*{\fill} Podpis mentorja: \phantom{prostor za podpis}

% \vspace{2cm}
% \hspace*{\fill} Podpis somentorja: \phantom{prostor za podpis}

\cleardoublepage
\pdfbookmark[1]{Povzetek}{abstract}

\begin{center}
\textbf{\naslovdela} \\[3mm]
\textsc{Povzetek} \\[2mm]
\end{center}
Tukaj napišemo povzetek vsebine. Sem sodi razlaga vsebine in ne opis tega, kako je delo
organizirano.

\vfill
\begin{center}
\textbf{English translation of the title} \\[3mm] % prevod slovenskega naslova dela
\textsc{Abstract}\\[2mm]
\end{center}

An abstract of the work is written here. This includes a short description of
the content and not the structure of your work.

\vfill\noindent
\textbf{Math.~Subj.~Class.~(2010):} oznake kot 74B05, 65N99, na voljo so na naslovu
\url{http://www.ams.org/msc/msc2010.html} \\[1mm]
\textbf{Ključne besede:} \kljucnebesede \\[1mm]
\textbf{Keywords:} \keywords

\cleardoublepage

\setcounter{page}{1}    % od sedaj naprej začni zopet z 1
\pagenumbering{arabic}  % in z arabskimi številkami

\section{Uvod}

\section{Osnovni pojmi}

Naj bo $M$ gladka mnogoterost. Za vsako točko $p \in M$ definiramo simetrično pozitivno-definitno bilinearno preslikavo $g_{p} \colon T_{p}M \times T_{p}M \to \R$, ki je gladko odvisna od $p$. Družino preslikav $g_{p}$ imenujemo \emph{Riemannova metrika} $g$ na mnogoterosti $M$.
Gladki mnogoterosti, opremljeni z Riemannovo metriko, pravimo \emph{Riemannova mnogoterost}.

Izkaže se, da vsaka mnogoterost razreda $\mathcal{C}^{r+1}$ premore Riemannovo metriko razreda $\mathcal{C}^{r}$.

Naj bo $M$ domena v $\R^{n}$ s koordinatami $x = (x_{1}, \dots, x_{n})$. Riemannova metrika na $M$ je tedaj oblike
\begin{align}
g_{p} = \sum_{i,j=1}^{n} g_{i,j}(p) dx_{i} dx_{j}, \quad p \in M,
\end{align}
kjer je $G(p) = [g_{i,j}(p)]_{i,j=1}^{n}$ simetrična pozitivno-definitna matrika za vse $p \in M$. Za tangentna vektorja $\xi = (\xi_{1}, \dots, \xi_{n}), \ \eta = (\eta_{1}, \dots, \eta_{n}) \in \R^{n}$ velja
\begin{align}
g_{p}(\xi, \eta) &= \sum_{i,j=1}^{n} g_{i,j}(p) \xi_{i} \eta_{j} = G(p) \xi \cdot \eta.
\end{align}

Vzemimo gladko imerzijo $x \colon M \to \widetilde{M}$ in Riemannovo metriko $\tilde{g}$ na $\widetilde{M}$. \emph{Pullback metric} $g = x^{*} \tilde{g}$ na $M$, definirano na paru tangentnih vektorjev $\xi, \eta \in T_{p}M$, podaja predpis
\begin{equation} \label{eq:pullback}
g_{p}(\xi, \eta) = \tilde{g}_{x(p)} (dx_{p}(\xi), dx_{p}(\eta)).
\end{equation}
Če je metrika $\tilde{g}$ razreda $\mathcal{C}^{r}$ in imerzija $x$ razreda $\mathcal{C}^{r+1}$, potem je tudi pullback metric $g = x^{*} \tilde{g}$ razreda $\mathcal{C}^{r}$.

Oglejmo si primer Riemannove metrike, ki jo bomo v nadaljevanju večkrat uporabili.
Na Evklidskem prostoru $\R^{n}$ s koordinatami $x = (x_{1}, \dots, x_{n})$ je definirana \emph{Evklidska metrika}
\begin{equation}
ds^2 = (dx_{1})^2 + \cdots + (dx_{n})^2,
\end{equation}
to je Riemannova metrika, ki ustreza identični matriki $I_{n}$. Naj bo $D$ domena v $\R^2$ in $x \colon D \to \R^{n}$ imerzija, podana s predpisom $x(u_1,u_2) = (x_{1}(u_1,u_2), \dots, x_{n}(u_1,u_2))$, $(u_1,u_2) \in D$. Pripadajoča metrika na $D$ je enaka
\begin{gather}
g = x^{*}ds^2 = g_{1,1}du_{1}^2 + g_{1,2}du_{1}du_{2} + g_{2,1}du_{2}du_{1} + g_{2,2}du_{2}^2, \\
g_{1,1} = |x_{u_1}|^2, \ g_{1,2} = g_{2,1} = x_{u_1} \cdot x_{u_2}, \ g_{2,2} = |x_{u_2}|^2
\end{gather}
in jo imenujemo \emph{prva fundamentalna forma} ploskve $M = x(D)$.

\begin{definicija}
\emph{Riemannova ploskev} je kompleksna mnogoterost kompleksne dimenzije $1$.
\end{definicija}

% Ukrivljenost
\subsection{Ukrivljenost}
%
Naj bo $M$ ploskev, $n \geq 3$ in $x \colon M \to \R^{n}$ imerzija razreda $\mathcal{C}^2$. Izberimo karto $(U, \phi)$ na $M$ in koordinate $u = (u_1, u_2) \in U$, tako da je zožitev $x|_{U} \colon U \to \R^{n}$ vložitev na orientabilno ploskev $S = x(U) \subset \R^{n}$. Izberimo točko $q \in U$ in označimo $p = x(q) \in S$. Naj bo $t \mapsto (u_1(t), u_2(t))$ parametrizacija vložene krivulje razreda $\mathcal{C}^2$ v $U$ ter $q = (u_1(t_0), u_2(t_0))$ za nek $t_0$. Vsaka krivulja, vložena v $S$, ki vsebuje točko $p$, je tedaj oblike
\begin{equation}
\alpha (t) = x(u_1(t), u_2(t)).
\end{equation}
Označimo z $s = s(t)$ ločno dolžino krivulje $\alpha$. Predpostavimo, da izbrana točka $p$ ustreza $p = \alpha(s_0) \in S$, označimo pripadajoč tangentni vektor $\nu = \alpha '(s_0) \in T_{p}S$ ter enotsko normalo $N \in N_{p}S$ v točki $p$. Količino
\begin{equation}
\kappa ^{N}(p, \nu) = \alpha ''(s_0) \cdot N
\end{equation}
imenujemo \emph{normalna ukrivljenost} ploskve $S$ v točki $p$ v tangentni smeri $\nu$ in smeri enotske normale $N$.

Oglejmo si preslikavo $ \kappa ^{N}(p, \cdot) \colon \{\nu \in T_{p}S ; \ |\nu|=1 \} \to \R$, $ \nu \mapsto \kappa ^{N}(p, \nu)$, kjer je $p \in S$ izbrana fiksna točka. Kot zvezna preslikava na kompaktni množici doseže minimalno in maksimalno vrednost,
\begin{align}
\kappa _{1}^{N}(p) = \min _{|\nu| = 1} \kappa ^{N}(p, \nu), \quad \kappa _{2}^{N}(p) = \max _{|\nu| = 1} \kappa ^{N}(p, \nu),
\end{align}
katerima pravimo \emph{glavni ukrivljenosti}.

\begin{definicija}
\begin{enumerate}
\item
\emph{Povprečna ukrivljenost} ploskve $S$ v točki $p$ in normalni smeri $N$ je povprečje glavnih ukrivljenosti,
\begin{equation} \label{eq:povp-ukr}
H^{N}(p) = \frac{1}{2} \left(\kappa _{1}^{N}(p) + \kappa _{2}^{N}(p) \right).
\end{equation}
\item
Njun produkt 
\begin{equation} \label{eq:Gauss-ukr}
K^{N}(p) = \kappa _{1}^{N}(p) \cdot \kappa _{2}^{N}(p)
\end{equation}
definira \emph{Gaussovo ukrivljenost} ploskve $S$ v točki $p$ in normalni smeri $N$.
\item
Projekcijo povprečne ukrivljenosti na normalno ravnino $N_{p}S$ v smeri tangentne ravnine $T_{p}S$ imenujemo \emph{vektor povprečne ukrivljenosti} ploskve $S$ v točki $p$ in označimo s $\textbf{\textup{H}}$. Enačba~\ref{eq:povp-ukr} se v tej notaciji glasi $H^{N}(p) = \textbf{\textup{H}} \cdot N$ za vsak $N \in N_{p}S$.
\end{enumerate}
\end{definicija}

\begin{lema}
Naj bo $x \colon M \to \R^{n}$ imerzija razreda $\mathcal{C}^2$. Tedaj velja
\begin{equation}
\Delta{x} = 2 \textbf{\textup{H}},
\end{equation}
kjer je $\Delta$ Laplaceov operator glede na Riemannovo metriko $g = x^{*}ds^2$ v točki $q \in M$ in $\textbf{\textup{H}}$ vektor povprečne ukrivljenosti v točki $p = x(q) \in S$.
\end{lema}

% Aproksimacijski izreki za Riemannove ploskve
\subsection{Aproksimacijski izreki za Riemannove ploskve}
%
\begin{izrek} [Rungejev aproksimacijski izrek za Riemannove ploskve]
Naj bo $M$ Riemannova ploskev in $K$ njena kompaktna podmnožica. 
Potem lahko vsako funkcijo $f$, ki je holomorfna na okolici $K$, aproksimiramo enakomerno na $K$ z meromorfnimi funkcijami $F$ na $M$ brez polov na $K$, ter s holomorfnimi funkcijami na $M$, če $K$ nima lukenj.
Funkcije $F$ lahko izberemo tako, da se z dano funkcijo $f$ na končni množici točk v $K$ ujemajo do izbranega končnega reda in da ima $F$ pole v podmnožici $E \subset M \backslash K$, kjer $E$ vsebuje točko v vsaki luknji množice $K$. 
\end{izrek}

\begin{definicija}
Naj bo $K$ kompaktna podmnožica Riemannove ploskve $M$. Njena \emph{holomorfna ogrinjača} je množica 
\begin{equation}
\widehat{K}_{\mathcal{O}(M)} = \{p \in M ; \ |f(p)| \leq \max_{K} |f| \ \text{za vse} \ f \in \mathcal{O}(M) \}.
\end{equation}
Če velja $K = \widehat{K}_{\mathcal{O}(M)}$, množico $K$ imenujemo \emph{Rungejeva množica}.
\end{definicija}

\begin{izrek} [Weierstrass-Florackov interpolacijski izrek]
Naj bo $M$ odprta Riemannova ploskev in $K$ njena Rungejeva podmnožica. Naj bo $A = \{ a_i \}_{i=1}^{\infty}$ zaprta diskretna podmnožica v $M$, $U$ odprta podmnožica $M$, tako da je $A \cup K \subset U$ in $f$ meromorfna funkcija na $U$ z ničlami in poli le v točkah množice $A$.
Potem za izbrane $\varepsilon > 0$ in števila $k_{i} \in \N$ obstaja meromorfna funkcija $F$ na $M$, za katero velja:
\begin{enumerate}
\item $|F(z) - f(z)| < \varepsilon$ za vse $z \in K$,
\item v točkah $a_i$ je razlika $F-f$ ničelna do reda $k_i$,
\item $F$ nima ničel in polov na $M \backslash A$.
\end{enumerate} 
\end{izrek}

\subsection{Variacija ploščine}
 %
\begin{definicija}
\begin{enumerate}
\item
Naj bo $M$ gladka kompaktna ploskev z robom, $n \geq 3$ in naj bo preslikava $x \colon M \to \R^{n}$ imerzija razreda $\mathcal{C}^2$. \emph{Variacija preslikave x s fiksnim robom} je 1-parametrična družina $\mathcal{C}^2$ preslikav 
\begin{gather}
x^{t} \colon M \to \R^{n},\ t \in (-\varepsilon, \varepsilon) \subset \R,
\end{gather}
če je $x^0 = x$ in za vse $t$ z intervala velja $x^{t} = x$ na $bM$.
%
\item
Naj bo $p \in M$. \emph{Variacijsko vektorsko polje} preslikave $x^{t}$ je vektorsko polje, definirano kot
\begin{equation}
E(p,t) = \frac{\partial{x^t(p)}}{\partial{t}} \in \R^{n}.
\end{equation}
\end{enumerate}
\end{definicija}

Opazimo, da je za dovolj majhne vrednosti $t$ preslikava $x^{t}$ imerzija.
Po definiciji je na $bM \times (-\varepsilon, \varepsilon)$ variacijsko vektorsko polje $E$ konstantno ničelno.

\begin{definicija}
Naj bo $x \colon M \to \R^{n}$ imerzija razreda $\mathcal{C}^2$. Ploskev $M$ imenujemo \emph{minimalna ploskev}, če za vsako kompaktno domeno $D \subset M$ z gladkim robom $bD$ in vsako gladko variacijo $x^{t}$ preslikave $x$ s fiksnim robom velja
\begin{equation} \label{eq:1-var-ploščine}
\frac{d}{dt} \Big|_{t=0} \text{Area}(x^{t}(D)) = 0.
\end{equation}
Ekvivalentno pravimo, da je minimalna ploskev stacionarna točka ploskovnega funkcionala $\text{Area} \colon D \to \R$.
\end{definicija}

Levo stran enakosti~\ref{eq:1-var-ploščine} imenujemo \emph{prva variacija ploščine} pri $t=0$. Slednjo z geometrijskimi lastnostmi preslikave $x$, natančneje ukrivljenostjo, povezuje \emph{prva variacijska formula} v naslednjem izreku. 

\begin{izrek} \label{izr:1-var-formula}
Naj bo $M$ gladka kompaktna ploskev z robom, $n \geq 3$ in $x \colon M \to \R^{n}$ imerzija razreda $\mathcal{C}^2$. Naj bo $E = \partial{x^{t}} / \partial{t}|_{t=0}$ variacijsko vektorsko polje preslikave $x^{t}$ pri $t=0$, $\textbf{\textup{H}}$ vektorsko polje povprečne ukrivljenosti preslikave $x$ in $dA$ ploščinski element glede na Riemannovo metriko $x^{*}ds^2$, definirano na $M$.
Potem za vsako gladko variacijo $x^{t} \colon M \to \R^{n}$ imerzije $x$ s fiksnim robom velja
\begin{equation} \label{eq:1-var-formula}
\frac{d}{dt} \Big|_{t=0} \text{Area}(x^{t}(M)) = -2 \int_{M} {E \cdot \textbf{\textup{H}} dA}.
\end{equation}
\end{izrek}

\begin{izrek}
Naj bo $x \colon M \to \R^{n}$ imerzija razreda $\mathcal{C}^2$. Ploskev $M$ je minimalna natanko tedaj, ko je na $M$ vektor povprečne ukrivljenosti $\textbf{\textup{H}}$ preslikave $x$ identično enak $0$.
\end{izrek}

S podobnimi tehnikami kot v dokazu Izreka~\ref{izr:1-var-formula} izpeljemo \emph{drugo variacijsko formulo}
\begin{equation}
\frac{d^2}{dt^2} \Big|_{t=0} \text{Area}(x^{t}(M)) = \int_{M} {(4|E|^{2} K^{E} + |\nabla{E}|^2) dA},
\end{equation}
kjer $K^{E} = K^{N}$ označuje Gaussovo ukrivljenost ploskve $M$.

\subsection{Weierstrassova formula}
%
Naj bo ploskev $M$ orientabilna in $x \colon M \to \R^{n}$ imerzija razreda $\mathcal{C}^2$. Potem preslikava $x$ določa enolično strukturo Riemannove ploskve na $M$, kjer je $x$ konformna imerzija. Zato bomo v nadaljevanju obravnavali Riemannove ploskve in pripadajoče konformne imerzije v Evklidski prostor.
Prvi rezultat, ki ga navajamo, opisuje ekvivalentne pogoje minimalnosti ploskve $M$.

\begin{izrek}
Naj bo $M$ odprta Riemannova ploskev, $n \geq 3$ in $x = (x_1, \dots , x_n) \colon M \to \R^{n}$ konformna imerzija razreda $\mathcal{C}^2$. Naslednje trditve so ekvivalentne:
\begin{enumerate}
	\item $x$ je minimalna ploskev.
	\item Vektorsko polje povprečne ukrivljenosti preslikave $x$ je ničelno, tj.~$\textbf{\textup{H}} = 0$.
	\item $x$ je harmonična, tj.~$\Delta{x} = 0$.
	\item 1-forma $ \partial{x} = (\partial{x_1}, \dots , \partial{x_n})$ z vrednostmi v $\C^{n}$ je holomorfna in velja
			\begin{equation}
			(\partial{x_1})^2 + \cdots + (\partial{x_n})^2 = 0.
			\end{equation}
	\item Naj bo $\theta$ holomorfna 1-forma na $M$, ki ni nikjer enaka $0$. Potem je preslikava $f = 2\partial{x} / \theta \colon M \to \C^{n}$ holomorfna z 				vrednostmi na \emph{ničelni kvadriki} 
			\begin{equation}		
			\textbf{\textup{A}} = \{ (z_1, \dots , z_n) \in \C^{n} ; z_{1}^{2} + \cdots + z_{n}^{2} = 0 \}.
			\end{equation}	
		Nadalje je Riemannova metrika na $M$, inducirana s konformno imerzijo $x$, enaka
			\begin{align}
			g &= x^{*} ds^2 = |dx_1|^2 + \cdots + |dx_n|^2 = 2 (|\partial{x_1}|^2 + \cdots |\partial{x_n}|^2).
			\end{align}			
\end{enumerate}
\end{izrek}

\begin{definicija}
Naj bo $x \colon M \to \R^{n}$ harmonična preslikava. Njen \emph{pretok} je homomorfizem grup $\textup{Flux}_{x} \colon H_{1} (M, \Z) \to \R^{n}$, definiran s predpisom 
\begin{equation}
\textup{Flux}_{x} ([C]) = \int_{C} {d^{c} x}.
\end{equation}
\end{definicija}

V definiciji pretoka je $[C] \in H_{1} (M, \Z),$ integral pa je odvisen le od homološkega razreda poti $C$, zato bomo v nadaljevanju pisali kar $\textup{Flux}_{x} (C)$.

\begin{definicija}
\begin{enumerate}
\item
Naj bo $M$ odprta Riemannova ploskev in $n \geq 3$. Holomorfno imerzijo $z = (z_{1}, \dots , z_{n}) \colon M \to \C^{n}$, za katero velja
$(\partial{z_{1}})^2 + \cdots + (\partial{z_{n}})^2 = 0$, imenujemo \emph{holomorfna ničelna krivulja} v $\C^{n}$.
\item
Naj bo $z = x + \imath y \colon M \to \C^{n}$ holomorfna ničelna krivulja. Njena realni del in imaginarni del, $x, y \colon M \to \R^{n}$ imenujemo \emph{konjugirani minimalni ploskvi}.
\item
Naj bo $t \in \R$. Predstavnike 1-parametrične družine $x^{t} = \Re{(e^{\imath t} z)} \colon M \to \R^{n}$ imenujemo \emph{asociirane minimalne ploskve} holomorfne ničelne krivulje $z$.
\end{enumerate}
\end{definicija}

\begin{izrek}[Weierstrassova predstavitev konformnih minimalnih ploskev in holomorfnih ničelnih krivulj]
Naj bo $n \geq 3$ in $M$ odprta Riemannova ploskev, na kateri definiramo holomorfno 1-formo $\Phi = (\phi_{1}, \dots , \phi_{n})$ z vrednsotmi v $\C^{n}$, ki je povsod neničelna, in zadošča 
\begin{enumerate}
\item $ \sum_{j=1}^{n} \phi_{j}^{2} = 0$,
\item $ \Re \int_{C} \Phi = 0 $ za vse $[C] \in H_{1} (M, \Z)$.
\end{enumerate}
Potem za poljuben izbor točk $p_0 \in M$ in $x_0 \in \R^{n}$ predpis $x \colon M \to \R^{n}$,
\begin{align} \label{eq:Wstrass-kmi}
x(p) = x_0 + \Re \int_{p_0}^{p} \Phi, \ p \in M,
\end{align}
podaja dobro definirano konformno minimalno imerzijo. Zanjo velja
\begin{align}
2 \partial{x} = \Phi \quad \text{in} \quad g = x^{*} ds^2 = |dx|^2 = \frac{1}{2} |\Phi|^2.
\end{align}
%
Če velja še
$ \int_{C} \Phi = 0 \ \text{za vse} \ [C] \in H_{1} (M, \Z) $,
potem za poljuben izbor točk $p_0 \in M$ in $z_0 \in \C^{n}$ predpis $z \colon M \to \C^{n}$,
\begin{align} \label{eq:Wstrass-hnk}
z(p) = z_0 + \int_{p_0}^{p} \Phi, \ p \in M,
\end{align}
podaja dobro definirano holomorfno ničelno krivuljo. Zanjo velja
\begin{align}
\partial{z} = \Phi \quad \text{in} \quad z^{*} ds^2 = |dz|^2 = |\partial{z}|^2 = |\Phi|^2.
\end{align}
\end{izrek}

\begin{opomba}
Vsaka konformna minimalna imerzija $x \colon M \to \R^{n}$ je oblike~\ref{eq:Wstrass-kmi} in vsaka holomorfna ničelna krivulja $z \colon M \to \C^{n}$ je oblike~\ref{eq:Wstrass-hnk}. Prav zato je Weierstrassova predstavitev elegantna metoda za konstrukcijo opisanih preslikav.
\end{opomba}

Če konformno minimalno imerzijo $x \colon M \to \R^{n}$ poznamo, potem pripadajočo povsod neničelno holomorfno 1-formo $\Phi = 2 \partial{x}$ z vrednostmi v $\C^{n}$ imenujemo \emph{Weierstrass data} preslikave $x$. 
Analogno, za holomorfno ničelno krivuljo $z \colon M \to \C^{n}$ pripadajočo 1-formo $\Phi = \partial{z} = dz$ imenujemo \emph{Weierstrass data} preslikave $z$.

\begin{definicija}
\emph{Jordanov lok} je pot v ravnini, ki je topološko izomorfna intervalu $[0,1]$.
\emph{Jordanova krivulja} je ravninska krivulja, ki je topološko ekvivalentna enotski krožnici.
\end{definicija}

\begin{definicija}
Naj bo $M$ gladka ploskev, $K$ končna unija paroma disjunktnih kompaktnih domen s kosoma zvezno odvedljivimi robovi v $M$ ter $E = S \setminus K^\circ$ unija končno mnogo paroma disjunktnih gladkih Jordanovih lokov in zaprtih Jordanovih krivulj, ki se dotikajo $K$ kvečjemu v svojih krajiščih in sekajo rob $K$ transverzalno. Kompaktno podmnožico v $M$ oblike $S = K \cup E$ imenujemo \emph{Admissible set}.
\end{definicija}

\begin{definicija}
Naj bo $M$ povezana odprta Riemannova ploskev ali kompaktna Riemannova ploskev z robom, na kateri je definirana povsod neničelna holomorfna 1-forma $\Theta$. Konformno minimalno imerzijo $x \colon M \to \R^{n}$ imenujemo:
\begin{enumerate}
\item \emph{flat}, če je slika $x(M)$ vsebovana v afini ravnini v $\R^{n}$; sicer pravimo, da je $x$ \emph{nonflat};
\item \emph{full}, če je preslikava $f = 2 \partial{x} / \Theta \colon M \to \textbf{\textup{A}}_{*}^{n-1}$ full, tj.\ $\C$-linearna ogrinjača slike $f(M)$ je enaka $\C^{n}$;
\item \emph{nedegenerirana}, če slika $x(M)$ ni vsebovana v nobeni hiperravnini v $\R^{n}$. 
\end{enumerate}
\end{definicija}

V dimenziji  $n=3$ za konformno minimalno imerzijo vsi zgornji pojmi sovpadajo. V višjih dimenzijah ($n \geq 4$) veljata implikaciji 
\[ \text{full} \ \Rightarrow \ \text{nedegenerirana} \ \Rightarrow \ \text{nonflat}. \]

\section{Izerki o aproksimaciji in interpolaciji minimalnih ploskev}


% Literatura:
% Primer navajanja na http://www.fmf.uni-lj.si/storage/24240/LiteraturaM.pdf,
% ampak bi moral stil poskrbeti za vse. Reference se uredijo po abecedi.
% Če nobena izbira izmed @book, @atricle,... ni ok, potem se lahko vse napiše v
% @misc pod note={} in deluje tako kot normalen LaTeX.
% Komentar v bib datoteki se naredi samo s parom { }
% Za urejanje literature avtor priporoča program Jabref, ki zna tudi avtomatsko
% okrajšati imena revij. Za pravilno sortiranje vnosov brez avtorja, uporabite
% polje key={ }, kot v primeru.
% V primeru napak ustvarite issue na GitHubu ali pišite na jure.slak@fmf.uni-lj.si.
\cleardoublepage                           % na desni strani
\phantomsection                            % da prav delujejo hiperlinki
\addcontentsline{toc}{section}{\bibname}   % dodajmo v kazalo
\bibliographystyle{fmf-sl}                 % uporabljen stil je v datoteki fmf-sl.bst, na voljo tudi angleška verzija
\bibliography{\literatura}                 % literatura je v datoteki, definirani na začetku
% TeXStudio zmede \ zgoraj, tako da lahko notri napišeš dejansko ime .bib datoteke, če ti
% ne delajo predlogi citatov.

% Za stvarno kazalo
\cleardoublepage                           % na desni strani
\phantomsection                            % da prav delujejo hiperlinki
\addcontentsline{toc}{section}{\indexname} % dodajmo v kazalo
\printindex

\end{document}
