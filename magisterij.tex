% !TeX spellcheck = sl_SI
% vim: set spell spelllang=sl:
% za preverjanje črkovanja, če se uporablja Texstudio ali vim
\documentclass[12pt,a4paper,twoside]{article}
\usepackage[utf8]{inputenc}  % pravilno razpoznavanje unicode znakov

% NASLEDNJE UKAZE USTREZNO POPRAVI
\newcommand{\program}{Matematika} % ime studijskega programa
\newcommand{\imeavtorja}{Tjaša Vrhovnik} % ime avtorja
\newcommand{\imementorja}{prof.~dr.~Franc Forstnerič} % akademski naziv in ime mentorja, uporabi poln naziv, prof.~dr.~, doc.~dr., ali izr.~prof.~dr.
\newcommand{\imesomentorja}{} % akademski naziv in ime somentorja, če ga imate
\newcommand{\naslovdela}{Minimalne ploskve}
\newcommand{\letnica}{2021} % letnica magistriranja
\newcommand{\opis}{}  % Opis dela v eni povedi. Ne sme vsebovati matematičnih simbolov v $ $.
\newcommand{\kljucnebesede}{} % ključne besede, ločene z \sep, da se PDF metapodatki prav procesirajo
\newcommand{\keywords}{} % ključne besede v angleščini
\newcommand{\organization}{Univerza v Ljubljani, Fakulteta za matematiko in fiziko} % fakulteta
\newcommand{\literatura}{literatura}  % pot do datoteke z literaturo (brez .bib končnice)
\newcommand{\sep}{, }  % separator med ključnimi besedami v besedilu
% KONEC PODATKOV

\usepackage{bibentry}         % za navajanje literature v programu dela s celim imenom
\nobibliography{\literatura}
\newcommand{\plancite}[1]{\item[\cite{#1}] \bibentry{#1}} % citiranje v programu dela

\usepackage{filecontents}  % za pisanje datoteke s PDF metapodatki
\usepackage{silence} \WarningFilter{latex}{Overwriting file}  % odstrani annoying warning o obstoju datoteke
% datoteka s PDF metapodatki, zgenerira se kot magisterij.xmpdata
\begin{filecontents*}{\jobname.xmpdata}
  \Title{\naslovdela}
  \Author{\imeavtorja}
  \Keywords{\kljucnebesede}
  \Subject{matematika}
  \Org{\organization}
\end{filecontents*}

\usepackage[a-1b]{pdfx}  % zgenerira PDF v tem PDF/A-1b formatu, kot zahteva knjižnica
\hypersetup{bookmarksopen, bookmarksdepth=3, colorlinks=true,
  linkcolor=black, anchorcolor=black, citecolor=black, filecolor=black,
  menucolor=black, runcolor=black, urlcolor=black, pdfencoding=auto,
  breaklinks=true, psdextra}

\usepackage[slovene]{babel}  % slovenščina
\usepackage[T1]{fontenc}     % naprednejše kodiranje fonta
\usepackage{amsmath,amssymb,amsfonts,amsthm} % matematični paketi
\usepackage{graphicx}     % za slike
\usepackage{emptypage}    % prazne strani so neoštevilčene, ampak so štete
\usepackage{units}        % fizikalne enote kot \unit[12]{kg} s polovico nedeljivega presledka, glej primer v kodi
\usepackage{makeidx}      % za stvarno kazalo, lahko zakomentiraš, če ne rabiš
\makeindex                % za stvarno kazalo, lahko zakomentiraš, če ne rabiš
% oblika strani
\usepackage[
  top=3cm,
  bottom=3cm,
  inner=3.5cm,      % margini za dvostransko tiskanje
  outer=2.5cm,
  footskip=40pt     % pozicija številke strani
]{geometry}

% VEČ ZANIMIVIH PAKETOV
% \usepackage{array}      % več možnosti za tabele
% \usepackage[list=true,listformat=simple]{subcaption}  % več kot ena slika na figure, omogoči slika 1a, slika 1b
% \usepackage[all]{xy}    % diagrami
% \usepackage{doi}        % za clickable DOI entrye v bibliografiji
% \usepackage{enumerate}     % več možnosti za sezname

% Za barvanje source kode
% \usepackage{minted}
% \renewcommand\listingscaption{Program}

% Za pisanje psevdokode
% \usepackage{algpseudocode}  % za psevdokodo
% \usepackage{algorithm}
% \floatname{algorithm}{Algoritem}
% \renewcommand{\listalgorithmname}{Kazalo algoritmov}

% DRUGI TVOJI PAKETI:
% tukaj

\setlength{\overfullrule}{50pt} % označi predlogo vrstico
\pagestyle{plain}               % samo številka strani na dnu, nobene glave / noge

% ukazi za matematična okolja
\theoremstyle{definition} % tekst napisan pokončno
\newtheorem{definicija}{Definicija}[section]
\newtheorem{primer}[definicija]{Primer}
\newtheorem{opomba}[definicija]{Opomba}
\newtheorem{aksiom}{Aksiom}
\newtheorem{dokaz}{Dokaz}

\theoremstyle{plain} % tekst napisan poševno
\newtheorem{lema}[definicija]{Lema}
\newtheorem{izrek}[definicija]{Izrek}
\newtheorem{trditev}[definicija]{Trditev}
\newtheorem{posledica}[definicija]{Posledica}

\numberwithin{equation}{section}  % števec za enačbe zgleda kot (2.7) in se resetira v vsakem poglavju

% Matematični ukazi
\newcommand{\R}{\mathbb R}
\newcommand{\N}{\mathbb N}
\newcommand{\Z}{\mathbb Z}
%\renewcommand{\C}{\mathbb C}
\newcommand{\C}{\mathbb C}
\newcommand{\Q}{\mathbb Q}

% \DeclareMathOperator{\tr}{tr}  % morda potrebuješ operator za sled ali kaj drugega?

% bold matematika znotraj \textbf{ }, tudi v naslovih, kot \omega spodaj
\makeatletter \g@addto@macro\bfseries{\boldmath} \makeatother

% Poimenuj kazalo slik kot ``Kazalo slik'' in ne ``Slike''
\addto\captionsslovene{
  \renewcommand{\listfigurename}{Kazalo slik}%
}

% če želiš, da se poglavja začnejo na lihih straneh zgoraj
% \let\oldsection\section
% \def\section{\cleardoublepage\oldsection}

%%%%%%%%%%%%%%%%%%%%%%%%%%%%%%%%%%%%%%%%%%
%%%%%%           DOCUMENT           %%%%%%
%%%%%%%%%%%%%%%%%%%%%%%%%%%%%%%%%%%%%%%%%%

\begin{document}

\pagenumbering{roman} % začnemo z rimskimi številkami
\thispagestyle{empty} % ampak na prvi strani ni številke

\noindent{\large
UNIVERZA V LJUBLJANI\\[1mm]
FAKULTETA ZA MATEMATIKO IN FIZIKO\\[5mm]
\program\ -- 2.~stopnja}
% ustrezno dopolni za IŠRM
\vfill

\begin{center}
  \large
  \imeavtorja\\[3mm]
  \Large
  \textbf{\MakeUppercase{\naslovdela}}\\[10mm]
  \large
  Magistrsko delo \\[1cm]
  Mentor: \imementorja \\[2mm] % ustrezno popravi spol
%   Somentor: \imesomentorja   % dodaj, če potrebno
\end{center}
\vfill

\noindent{\large Ljubljana, \letnica}

\cleardoublepage

%% sem pride IZJAVA O AVTORSTVU  -- SE NATISNE V VIS

% zahvala
\pdfbookmark[1]{Zahvala}{zahvala} %
\section*{Zahvala}
% end zahvala -- izbriši vse med zahvala in end zahvala, če je ne rabiš

\cleardoublepage

\pdfbookmark[1]{\contentsname}{kazalo-vsebine}
\tableofcontents

% list of figures
% \cleardoublepage
% \pdfbookmark[1]{\listfigurename}{kazalo-slik}
% \listoffigures
% end list of figures

\cleardoublepage

\section*{Program dela}
\addcontentsline{toc}{section}{Program dela} % dodajmo v kazalo

\section*{Osnovna literatura}
Literatura mora biti tukaj posebej samostojno navedena (po pomembnosti) in ne
le citirana. V tem razdelku literature ne oštevilčimo po svoje, ampak uporabljamo
okolje itemize in ukaz plancite, saj je celotna literatura oštevilčena na koncu.
\begin{itemize}
  \plancite{lebedev2009introduction}
  \plancite{gurtin1982introduction}
  \plancite{zienkiewicz2000finite}
  \plancite{STtemplate}
\end{itemize}

\vspace{2cm}
\hspace*{\fill} Podpis mentorja: \phantom{prostor za podpis}

% \vspace{2cm}
% \hspace*{\fill} Podpis somentorja: \phantom{prostor za podpis}

\cleardoublepage
\pdfbookmark[1]{Povzetek}{abstract}

\begin{center}
\textbf{\naslovdela} \\[3mm]
\textsc{Povzetek} \\[2mm]
\end{center}
Tukaj napišemo povzetek vsebine. Sem sodi razlaga vsebine in ne opis tega, kako je delo
organizirano.

\vfill
\begin{center}
\textbf{English translation of the title} \\[3mm] % prevod slovenskega naslova dela
\textsc{Abstract}\\[2mm]
\end{center}

An abstract of the work is written here. This includes a short description of
the content and not the structure of your work.

\vfill\noindent
\textbf{Math.~Subj.~Class.~(2010):} oznake kot 74B05, 65N99, na voljo so na naslovu
\url{http://www.ams.org/msc/msc2010.html} \\[1mm]
\textbf{Ključne besede:} \kljucnebesede \\[1mm]
\textbf{Keywords:} \keywords

\cleardoublepage

\setcounter{page}{1}    % od sedaj naprej začni zopet z 1
\pagenumbering{arabic}  % in z arabskimi številkami

% Uvod
\section{Uvod}

% Osnovni pojmi
\section{Osnovni pojmi}

% Mnogoterosti
\subsection{Mnogoterosti}

\begin{definicija}
Naj bo $n \in \N$. Topološki prostor $M$ z lastnostmi:
\begin{enumerate}
\item $M$ je Hausdorffov,
\item $M$ je 2-števen,
\item $M$ je lokalno evklidski prostor dimenzije $n$ (za vsak $p \in M$ obstajata odprta okolica $U \subset M$ in homeomorfizem $\Phi \colon U \to V \subset \R^{n}$, kjer je $V$ odprta množica),
\end{enumerate}
imenujemo \emph{topološka mnogoterost} dimenzije $n$.
\end{definicija}

Na topološki mnogoterosti $M$ dimenzije $n$ definiramo \emph{atlas} $\mathcal{U} = \{ (U_{i}, \Phi_{i}) ; \ i \in I \}$ kot družino parov $(U_{i}, \Phi_{i})$, kjer je $\{ U_{i} \}_{i \in I}$ odprto pokritje mnogoterosti $M$, preslikave $\Phi_{i} \colon U_{i} \to \Phi_{i}(U_{i}) \subset \R^{n}$ pa so homeormorfizmi za vse $i$. Par $(U_{i}, \Phi_{i})$ imenujemo \emph{lokalna karta}.
Vzemimo lokalni karti $(U_{i}, \Phi_{i})$ in $(U_{j}, \Phi_{j})$, $i \neq j$, za kateri velja $U_{ij}=U_{i} \cap U_{j} \neq \emptyset$. Difeomorfizmu $\Phi_{ij} = \Phi_{j} \circ \Phi_{i}^{-1} \colon \Phi_{i}(U_{ij}) \to \Phi_{j}(U_{ij})$ med odprtima podmnožicama $\R^{n}$ pravimo \emph{prehodna preslikava} med lokalnima kartama. Atlas je razreda $\mathcal{C}^{r}$ za $r \geq 1$, kadar so prehodne preslikave med vsemi lokalnimi kartami difeomorfizmi razreda $\mathcal{C}^{r}$. V tem primeru rečemo, da je $M$ \emph{mnogoterost razreda} $\mathcal{C}^{r}$.
V posebnem gladek atlas določa gladko mnogoterost.

\begin{definicija}
Naj bo $X$ mnogoterost razreda $\mathcal{C}^{r}$ razsežnosti $\dim X = n$ in $M \subset X$ njena podmnožica. Če za vsako točko $p \in M$ obstaja lokalna karta $(U, \Phi)$ glede na atlas $\mathcal{U}$ mnogoterosti $X$, tako da je preslikava $\Phi \colon U \to V \subset \mathbb{R}^{n}$ homeomorfizem in velja $\Phi (M \cap U) = V \cap (\mathbb{R}^{m} \times \{0\}^{n-m})$, potem $M$ imenujemo \emph{podmnogoterost} razreda $\mathcal{C}^{r}$ razsežnosti $\dim M = m$.
\end{definicija}

Definirati želimo še tangentni prostor mnogoterosti. Naj bo $M$ gladka mnogoterost in izberimo atlas $\mathcal{U} = \{ (U_{i}, \Phi_{i}) ; \ i \in I \}$ na njej. Naj bo točka $p \in U_{i} \subset M$ za nek indeks $i$.  Gladki krivulji\footnote{Krivulja $\gamma_{j}$ je gladka, če je preslikava $ \Phi_{i} \circ \gamma_{j} \colon (-\varepsilon, \varepsilon) \to \mathbb{R}^{n}$, $j=1,2,$ gladka v običajnem smislu.} 
$\gamma_{1}, \gamma_{2} \colon (-\varepsilon, \varepsilon) \to M$ sta \emph{ekvivalentni}, če izpolnjujeta pogoja
$\gamma_{1}(0) = \gamma_{2}(0) = p$ in $ \frac{d}{dt}|_{t=0} \Phi_{i}(\gamma_{1}(t)) =  \frac{d}{dt}|_{t=0} \Phi_{i}(\gamma_{2}(t))$ za vse $t \in (-\varepsilon, \varepsilon)$. Ekvivalenco krivulj označimo z $\gamma_{1} \sim \gamma_{2}$\footnote{Relacija $\sim$ je ekvivalenčna relacija.}.

\begin{definicija}
Naj bo $M$ mnogoterost in $p \in M$ točka na njej. \emph{Tangentni vektor} $v_{p}$ na $M$ v točki $p$ ustreza ekvivalenčnemu razredu $[\gamma]$ krivulje $\gamma \colon (-\varepsilon, \varepsilon) \to M$, za katero velja $\gamma (0) = p$.

Unija vseh tangentnih vektorjev na $M$ v točki $p$ določa \emph{tangentni prostor} $T_{p}M$ mnogoterosti $M$ v točki $p$.
\end{definicija}

Naj bosta $M$ in $N$ mnogoterosti dimenzij $\dim M = m$, $\dim N = n$ ($m, n \in \N$). Naj bo $r \geq 0$. Pravimo, da je zvezna preslikava $f \colon M \to N$ \emph{razreda $\mathcal{C}^{r}$ v točki} $p \in M$, če obstajata taki $\mathcal{C}^{r}$ karti $(U, \Phi)$ na $M$ v okolici točke $p \in M$ in $(V, \Psi)$ na $N$ v okolici točke $f(p) \in N$, da je preslikava $F = \psi \circ f \circ \Phi^{-1}$ razreda $\mathcal{C}^{r}$ v okolici točke $\Phi(p)$.
Če to velja za poljubno točko $p \in M$, je $f$ \emph{razreda $\mathcal{C}^{r}$}; pišemo $f \in  \mathcal{C}^{r}(M,N)$.

Vzemimo gladki (oz.~razreda $\mathcal{C}^{r}$, $r \geq 1$) mnogoterosti $M$ in $N$ ter točko $p \in M$. \emph{Diferencial} gladke (oz.~razreda $\mathcal{C}^{r}$) preslikave $f \colon M \to N$ je linearna preslikava $df \colon T_{p}M \to T_{f(p)}N$, definirana s predpisom
\[ (df_{p})[\gamma] = [f \circ \gamma]. \]

\begin{definicija}
Naj bo $f \colon M \to N$ gladka preslikava med gladkima mnogoterostima. Preslikava $f$ se imenuje 
\begin{enumerate}
\item \emph{imerzija}, če je njen diferencial $df_{p}$ injektiven v vsaki točki $p \in M$;
\item \emph{submerzija}, če je njen diferencial $df_{p}$ surjektiven v vsaki točki $p \in M$;
\item \emph{vložitev}, če je $f$ injektivna preslikava in slika $f(M) \subset N$ podmnogoterost.
\end{enumerate}
\end{definicija}

\begin{opomba}
Z uporabo izreka o implicitni preslikavi dokažemo naslednje: Če je $f \colon M \to N$ submerzija v okolici točke $p \in U$ ($U \subset M$ odprta), potem je praslika $f^{-1}f(p))$ podmnogoterost v $M$ razsežnosti $\dim M - \dim N$.
\end{opomba}

% Riemannova metrika & mnogoterost
\begin{definicija}
Naj bo $M$ gladka mnogoterost. Za vsako točko $p \in M$ definiramo simetrično pozitivno-definitno bilinearno preslikavo $g_{p} \colon T_{p}M \times T_{p}M \to \R$, ki je gladko odvisna od $p$. Družino preslikav $g_{p}$ imenujemo \emph{Riemannova metrika} $g$ na mnogoterosti $M$.
Gladki mnogoterosti, opremljeni z Riemannovo metriko, pravimo \emph{Riemannova mnogoterost}.
\end{definicija}

Izkaže se, da vsaka mnogoterost razreda $\mathcal{C}^{r+1}$ premore Riemannovo metriko razreda $\mathcal{C}^{r}$.

Naj bo $M$ domena v $\R^{n}$ s koordinatami $x = (x_{1}, \dots, x_{n})$. Riemannova metrika na $M$ je tedaj oblike
\begin{align}
g_{p} = \sum_{i,j=1}^{n} g_{i,j}(p) dx_{i} dx_{j}, \quad p \in M,
\end{align}
kjer je $G(p) = [g_{i,j}(p)]_{i,j=1}^{n}$ simetrična pozitivno-definitna matrika za vse $p \in M$. Za tangentna vektorja $\xi = (\xi_{1}, \dots, \xi_{n}), \ \eta = (\eta_{1}, \dots, \eta_{n}) \in \R^{n}$ velja
\begin{align}
g_{p}(\xi, \eta) &= \sum_{i,j=1}^{n} g_{i,j}(p) \xi_{i} \eta_{j} = G(p) \xi \cdot \eta.
\end{align}

Vzemimo gladko imerzijo $x \colon M \to \widetilde{M}$ in Riemannovo metriko $\tilde{g}$ na $\widetilde{M}$. \emph{Povlečeno metriko} $g = x^{*} \tilde{g}$ na $M$, definirano na paru tangentnih vektorjev $\xi, \eta \in T_{p}M$, podaja predpis
\begin{equation} \label{eq:pullback-metrika}
g_{p}(\xi, \eta) = \tilde{g}_{x(p)} (dx_{p}(\xi), dx_{p}(\eta)).
\end{equation}
Če je metrika $\tilde{g}$ razreda $\mathcal{C}^{r}$ in imerzija $x$ razreda $\mathcal{C}^{r+1}$, potem je tudi povlečena metrika $g = x^{*} \tilde{g}$ razreda $\mathcal{C}^{r}$.

\begin{primer}[Prva fundamentalna forma]
Oglejmo si primer Riemannove metrike na realnem $n$-razsežnem Evklidskem prostoru $\mathbb{R}^{n}$.
Če izberemo standardne koordinate $x = (x_{1}, \dots, x_{n})$, \emph{Evklidsko metriko} definira predpis
\begin{equation}
ds^2 = (dx_{1})^2 + \cdots + (dx_{n})^2;
\end{equation}
to je Riemannova metrika, ki ustreza identični matriki $I_{n}$. Naj bo $D$ domena v $\R^2$ in $x \colon D \to \R^{n}$ imerzija, podana s predpisom $x(u_1,u_2) = (x_{1}(u_1,u_2), \dots, x_{n}(u_1,u_2))$ za $(u_1,u_2) \in D$. Pripadajoča metrika na $D$ je enaka
\begin{gather}
g = x^{*}ds^2 = g_{1,1}du_{1}^2 + g_{1,2}du_{1}du_{2} + g_{2,1}du_{2}du_{1} + g_{2,2}du_{2}^2, \\
g_{1,1} = |x_{u_1}|^2, \ g_{1,2} = g_{2,1} = x_{u_1} \cdot x_{u_2}, \ g_{2,2} = |x_{u_2}|^2
\end{gather}
in jo imenujemo \emph{prva fundamentalna forma} ploskve $M = x(D)$.
\end{primer}

\begin{definicija}
\emph{Riemannova ploskev} je kompleksna mnogoterost kompleksne dimenzije $1$.
\end{definicija}

% Ukrivljenost
\subsection{Ukrivljenost}
%
Naj bo $M$ ploskev, $n \geq 3$ in $x \colon M \to \R^{n}$ imerzija razreda $\mathcal{C}^2$. Izberimo karto $(U, \phi)$ na $M$ in koordinate $u = (u_1, u_2) \in U$, tako da je zožitev $x|_{U} \colon U \to \R^{n}$ vložitev na orientabilno ploskev $S = x(U) \subset \R^{n}$. Izberimo točko $q \in U$ in označimo $p = x(q) \in S$. Naj bo $t \mapsto (u_1(t), u_2(t))$ parametrizacija vložene krivulje razreda $\mathcal{C}^2$ v $U$ ter $q = (u_1(t_0), u_2(t_0))$ za nek $t_0$. Vsaka krivulja, vložena v $S$, ki vsebuje točko $p$, je tedaj oblike
\begin{equation}
\alpha (t) = x(u_1(t), u_2(t)).
\end{equation}
Označimo z $s = s(t)$ ločno dolžino krivulje $\alpha$. Predpostavimo, da izbrana točka $p$ ustreza $p = \alpha(s_0) \in S$, označimo pripadajoč tangentni vektor $\nu = \alpha '(s_0) \in T_{p}S$ ter enotsko normalo $N \in N_{p}S$ v točki $p$. Količino
\begin{equation}
\kappa ^{N}(p, \nu) = \alpha ''(s_0) \cdot N
\end{equation}
imenujemo \emph{normalna ukrivljenost} ploskve $S$ v točki $p$ v tangentni smeri $\nu$ in smeri enotske normale $N$.

Oglejmo si preslikavo $ \kappa ^{N}(p, \cdot) \colon \{\nu \in T_{p}S ; \ |\nu|=1 \} \to \R$, $ \nu \mapsto \kappa ^{N}(p, \nu)$, kjer je $p \in S$ izbrana fiksna točka. Kot zvezna preslikava na kompaktni množici doseže minimalno in maksimalno vrednost,
\begin{align}
\kappa _{1}^{N}(p) = \min _{|\nu| = 1} \kappa ^{N}(p, \nu), \quad \kappa _{2}^{N}(p) = \max _{|\nu| = 1} \kappa ^{N}(p, \nu),
\end{align}
katerima pravimo \emph{glavni ukrivljenosti}.

\begin{definicija}
\begin{enumerate}
\item
\emph{Povprečna ukrivljenost} ploskve $S$ v točki $p$ in normalni smeri $N$ je povprečje glavnih ukrivljenosti,
\begin{equation} \label{eq:povp-ukr}
H^{N}(p) = \frac{1}{2} \left(\kappa _{1}^{N}(p) + \kappa _{2}^{N}(p) \right).
\end{equation}
\item
Njun produkt 
\begin{equation} \label{eq:Gauss-ukr}
K^{N}(p) = \kappa _{1}^{N}(p) \cdot \kappa _{2}^{N}(p)
\end{equation}
definira \emph{Gaussovo ukrivljenost} ploskve $S$ v točki $p$ in normalni smeri $N$.
\item
Projekcijo povprečne ukrivljenosti na normalno ravnino $N_{p}S$ v smeri tangentne ravnine $T_{p}S$ imenujemo \emph{vektor povprečne ukrivljenosti} ploskve $S$ v točki $p$ in označimo s $\textbf{\textup{H}}$. Enačba~\ref{eq:povp-ukr} se v tej notaciji glasi $H^{N}(p) = \textbf{\textup{H}} \cdot N$ za vsak $N \in N_{p}S$.
\end{enumerate}
\end{definicija}

\begin{lema}
Naj bo $x \colon M \to \R^{n}$ imerzija razreda $\mathcal{C}^2$. Tedaj velja
\begin{equation}
\Delta{x} = 2 \textbf{\textup{H}},
\end{equation}
kjer je $\Delta$ Laplaceov operator glede na Riemannovo metriko $g = x^{*}ds^2$ v točki $q \in M$ in $\textbf{\textup{H}}$ vektor povprečne ukrivljenosti v točki $p = x(q) \in S$.
\end{lema}

% Vektorska polja
\subsection{Vektorska polja}
%
\begin{definicija}
Naj bo $r \geq 1$ ter $E$ in $B$ mnogoterosti razreda $\mathcal{C}^{r}$.
Surjektivno preslikavo $\pi \colon E \to B$ imenujemo realen \emph{vektorski sveženj} ranga $n$ in razreda $\mathcal{C}^{r}$, če
\begin{enumerate}
\item je vsako vlakno $\pi^{-1}(b) = E_{b}$, $b \in B$, $n$-razsežen realen vektorski prostor: $E_{b} \cong \R^{n}$,
\item za vsak $b \in B$ obstajata okolica $b \in U \subset B$ in difeomorfizem $\tau \colon E|_{U} \to U \times \R^{n}$ razreda $\mathcal{C}^{r}$, tako da je za vsak $x \in U$ preslikava $ \tau_{x} \colon E_{x} \to \{x\} \times \R^{n} $ linearni izomorfizem. Preslikavi $\tau_{x}$ pravimo \emph{lokalna trivializacija}.
\end{enumerate}
Če ima vlakno strukturo kompleksnega vektorskega prostora, na ustreznih mestih v definiciji zamenjamo $\R^{n}$ s $\C^{n}$ -- v tem primeru dobimo kompleksen vektorski sveženj.
\end{definicija}

\begin{definicija}
\emph{Prerez} vektorskega svežnja $\pi \colon E \to B$ je preslikava $s \colon B \to E$, za katero velja $\pi \circ s = id_{B}$.
Ekvivalentno, za vsak $b \in B$ je $s(b) \in \pi^{-1}(b) = E_{b}$, torej prerez vsako točko baznega prostora slika v točko v vlaknu nad $b$.
\end{definicija}

Omenimo poseben primer vektorskega svežnja, ki bo pomemben v nadaljevanju. Naj bo $X$ mnogoterost razreda $\mathcal{C}^{r}$ z $r \geq 1$. Njen \emph{tangentni sveženj} je disjunktna unija tangentnih prostorov na $X$ v točkah $x \in X$:
\[ TX = \bigsqcup_{x \in X} T_{x}X. \]
Tangentni sveženj je vektorski sveženj ranga $n = \dim X$ in razreda $\mathcal{C}^{r-1}$.

\begin{definicija}
Naj bo $X$ mnogoterost razreda $\mathcal{C}^{r}$, kjer je $r \geq 1$. Prerez njenega tangentnega svežnja, to je preslikava 
\begin{align*}
V \colon X \to TX, \ V(x) = V_{x} \in T_{x}X, \ x \in X,
\end{align*}
se imenuje \emph{vektorsko polje} na X. Prostor vektorskih polj na $X$ označimo z $\Gamma (X)$.
\end{definicija}

\begin{definicija}
Naj bo $V$ vektorsko polje na mnogoterosti $X$ in $x \in X$ točka, v kateri je vektorsko polje neničelno. Pot $\gamma_{x} \colon (-\varepsilon, \varepsilon) \subset \R \to X$ razreda $\mathcal{C}^{1}$ je \emph{integralna krivulja} vektorskega polja $V$ skozi $x$, če je $\gamma_{x}(0) = x$ in
\begin{align*}
\dot{\gamma}_{x} (t) = V(\gamma _{x}(t)) \in T_{\gamma _{x}(t)}X, \ t \in (-\varepsilon, \varepsilon).
\end{align*}

Naj bo $U \subset X$ odprta množica, na kateri je vektorsko polje $V$ neničelno. \emph{Tok vektorskega polja} $V$ na $U$ je 1-parametrična družina preslikav
$ \Phi_{t} \colon U \to \Phi_{t}(U),$ definiranih s predpisi $\Phi_{t}(x) = \gamma_{x}(t)$.
\end{definicija}

Vektorsko polje $V$ lahko v lokalnih koordinatah $x = (x_{1}, \dots, x_{n})$ na odprti podmnožici $U \subset X$ zapišemo kot 
\begin{align}
V(m) = \sum_{i=1}^{n} V_{i}(m) \frac{\partial}{\partial x_{i}},
\end{align}
kjer so $V_{i}$ realne funkcije na $U$, diferenciali $\frac{\partial}{\partial x_{i}}$ pa v vsaki točki $m \in U$ sestavljajo bazo tangentnega prostora $T_{m}X$.
Pot $\gamma (t) = (\gamma_{1}(t), \dots, \gamma_{n}(t))$ na $X$ je po definiciji integralna krivulja vektorskega polja $V$ natanko takrat, ko zadošča enakosti 
\begin{align*}
\dot{\gamma}(t) = \sum_{i=1}^{n} V_{i}(\gamma(t)) \frac{\partial}{\partial x_{i}}.
\end{align*}
Rešujemo sistem $n$ navadnih diferencialnih enačb ($i \in \{ 1, \dots , n \}$)
\begin{align*}
\dot{\gamma}_{i}(t) = V_{i}(\gamma_{1}(t), \dots, \gamma_{n}(t)),
\end{align*}
katerega lokalna rešitev je tok vektorskega polja $V$ na $X$, $\Phi_{t}(m)$. Po eksistenčnem izreku za navadne diferencialne enačbe lokalna rešitev vedno obstaja.

Zanimajo nas duali tangentnih prostorov ter prerezi pripradajočih svežnjev.

\begin{definicija}
Naj bo $X$ gladka mnogoterost. Dualni sveženj njenega tangentnega svežnja imenujemo \emph{kotangentni sveženj}
\begin{align}
T^{*}X = (TX)^{*} = \bigsqcup_{x \in X} T_{x}^{*}X.
\end{align}
Tu je $T_{x}^{*}X$ \emph{kotangentni prostor} mnogoterosti $X$ v točki $x \in X$, ki je sestavljen iz linearnih funkcionalov $T_{x}^{*}X \to \R$.
\emph{(Diferencialna) 1-forma} na mnogoterosti $X$ je prerez $\alpha \colon X \to T^{*}X$ kotangentnega svežnja. Prostor diferencialnih 1-form na $X$ označimo z $\Omega ^{1}(X)$.
\end{definicija}

Podobno kot vektorska polja lahko tudi 1-forme predstavimo lokalno. Naj bo $U$ odprta podmnožica v $X$ z lokalnimi koordinatami $x = (x_{1}, \dots, x_{n})$. Če so $a_{i}$ realne funkcije na $U$ in $dx_{i}$ diferenciali koordinatnih funkcij, ki v vsaki točki $m \in U$ tvorijo bazo kotangentnega prostora $T_{m}^{*}X$, potem ima poljubna 1-forma na $U$ obliko
\begin{align}
\alpha (m) = \sum_{i=1}^{n} a_{i}(m) dx_{i}.
\end{align}
Baza kotangentnega prostora je dualna bazi tangentnega prostora; natančneje, 
\begin{align*}
dx_{i}(m) \left(\frac{\partial}{\partial x_{j}} (m) \right) = \delta _{ij}.
\end{align*}

% Aproksimacijski izreki za Riemannove ploskve
\subsection{Aproksimacijski izreki za Riemannove ploskve}
%
\begin{izrek} [Rungejev aproksimacijski izrek za Riemannove ploskve]
Naj bo $M$ Riemannova ploskev in $K$ njena kompaktna podmnožica. 
Potem lahko vsako funkcijo $f$, ki je holomorfna na okolici $K$, aproksimiramo enakomerno na $K$ z meromorfnimi funkcijami $F$ na $M$ brez polov na $K$, ter s holomorfnimi funkcijami na $M$, če $K$ nima lukenj.
Funkcije $F$ lahko izberemo tako, da se z dano funkcijo $f$ na končni množici točk v $K$ ujemajo do izbranega končnega reda in da ima $F$ pole v podmnožici $E \subset M \backslash K$, kjer $E$ vsebuje točko v vsaki luknji množice $K$. 
\end{izrek}

\begin{definicija}
Naj bo $K$ kompaktna podmnožica Riemannove ploskve $M$. Njena \emph{holomorfna ogrinjača} je množica 
\begin{equation}
\widehat{K}_{\mathcal{O}(M)} = \{p \in M ; \ |f(p)| \leq \max_{K} |f| \ \text{za vse} \ f \in \mathcal{O}(M) \}.
\end{equation}
Če velja $K = \widehat{K}_{\mathcal{O}(M)}$, množico $K$ imenujemo \emph{Rungejeva množica}.
\end{definicija}

\begin{izrek} [Bishop-Mergelyanov aproksimacijski izrek] \label{izr:Bishop-Mergelyan}
Naj bo $M$ odprta Riemannova ploskev in $K$ njena kompaktna podmnožica brez lukenj ($K$ je Rungejeva v $M$). Potem lahko vsako funkcijo v $\mathcal{A}(K)$ aproksimiramo enakomerno na $K$ s funkcijami v $\mathcal{O}(M)$.
\end{izrek}

\begin{izrek} [Weierstrass-Florackov interpolacijski izrek]
Naj bo $M$ odprta Riemannova ploskev in $K$ njena Rungejeva podmnožica. Naj bo $A = \{ a_i \}_{i=1}^{\infty}$ zaprta diskretna podmnožica v $M$, $U$ odprta podmnožica $M$, tako da je $A \cup K \subset U$ in $f$ meromorfna funkcija na $U$ z ničlami in poli le v točkah množice $A$.
Potem za izbrane $\varepsilon > 0$ in števila $k_{i} \in \N$ obstaja meromorfna funkcija $F$ na $M$, za katero velja:
\begin{enumerate}
\item $|F(z) - f(z)| < \varepsilon$ za vse $z \in K$,
\item v točkah $a_i$ je razlika $F-f$ ničelna do reda $k_i$,
\item $F$ nima ničel in polov na $M \backslash A$.
\end{enumerate} 
\end{izrek}

% Variacija ploščine
\subsection{Variacija ploščine}
 %
\begin{definicija}
\begin{enumerate}
\item
Naj bo $M$ gladka kompaktna ploskev z robom, $n \geq 3$ in naj bo preslikava $x \colon M \to \R^{n}$ imerzija razreda $\mathcal{C}^2$. \emph{Variacija preslikave x s fiksnim robom} je 1-parametrična družina $\mathcal{C}^2$ preslikav 
\begin{gather}
x^{t} \colon M \to \R^{n},\ t \in (-\varepsilon, \varepsilon) \subset \R,
\end{gather}
če je $x^0 = x$ in za vse $t$ z intervala velja $x^{t} = x$ na $bM$.
%
\item
Naj bo $p \in M$. \emph{Variacijsko vektorsko polje} preslikave $x^{t}$ je vektorsko polje, definirano kot
\begin{equation}
E(p,t) = \frac{\partial{x^t(p)}}{\partial{t}} \in \R^{n}.
\end{equation}
\end{enumerate}
\end{definicija}

Opazimo, da je za dovolj majhne vrednosti $t$ preslikava $x^{t}$ imerzija.
Po definiciji je na $bM \times (-\varepsilon, \varepsilon)$ variacijsko vektorsko polje $E$ konstantno ničelno.

\begin{definicija}
Naj bo $x \colon M \to \R^{n}$ imerzija razreda $\mathcal{C}^2$. Ploskev $M$ imenujemo \emph{minimalna ploskev}, če za vsako kompaktno domeno $D \subset M$ z gladkim robom $bD$ in vsako gladko variacijo $x^{t}$ preslikave $x$ s fiksnim robom velja
\begin{equation} \label{eq:1-var-ploščine}
\frac{d}{dt} \Big|_{t=0} \text{Area}(x^{t}(D)) = 0.
\end{equation}
Ekvivalentno pravimo, da je minimalna ploskev stacionarna točka ploskovnega funkcionala $\text{Area}$.
\end{definicija}

Levo stran enakosti~\ref{eq:1-var-ploščine} imenujemo \emph{prva variacija ploščine} pri $t=0$. Slednjo z geometrijskimi lastnostmi preslikave $x$, natančneje ukrivljenostjo, povezuje \emph{prva variacijska formula} v naslednjem izreku. 

\begin{izrek} \label{izr:1-var-formula}
Naj bo $M$ gladka kompaktna ploskev z robom, $n \geq 3$ in $x \colon M \to \R^{n}$ imerzija razreda $\mathcal{C}^2$. Naj bo $E = \partial{x^{t}} / \partial{t}|_{t=0}$ variacijsko vektorsko polje preslikave $x^{t}$ pri $t=0$, $\textbf{\textup{H}}$ vektorsko polje povprečne ukrivljenosti preslikave $x$ in $dA$ ploščinski element glede na Riemannovo metriko $x^{*}ds^2$, definirano na $M$.
Potem za vsako gladko variacijo $x^{t} \colon M \to \R^{n}$ imerzije $x$ s fiksnim robom velja
\begin{equation} \label{eq:1-var-formula}
\frac{d}{dt} \Big|_{t=0} \text{Area}(x^{t}(M)) = -2 \int_{M} {E \cdot \textbf{\textup{H}} dA}.
\end{equation}
\end{izrek}

\begin{izrek}
Naj bo $x \colon M \to \R^{n}$ imerzija razreda $\mathcal{C}^2$. Ploskev $M$ je minimalna natanko tedaj, ko je na $M$ vektor povprečne ukrivljenosti $\textbf{\textup{H}}$ preslikave $x$ identično enak $0$.
\end{izrek}

S podobnimi tehnikami kot v dokazu Izreka~\ref{izr:1-var-formula} izpeljemo \emph{drugo variacijsko formulo}
\begin{equation}
\frac{d^2}{dt^2} \Big|_{t=0} \text{Area}(x^{t}(M)) = \int_{M} {(4|E|^{2} K^{E} + |\nabla{E}|^2) dA},
\end{equation}
kjer $K^{E} = K^{N}$ označuje Gaussovo ukrivljenost ploskve $M$.

% Weierstrassova formula
\subsection{Weierstrassova formula}
%
Naj bosta $(M,g)$ in $(\widetilde{M},\tilde{g})$ Riemannovi mnogoterosti z $\dim(M) \leq \dim(\widetilde{M})$.
Imerzija $x \colon (M,g) \to (\widetilde{M}, \tilde{g})$ se imenuje \emph{konformna}, če ohranja kote.
Z drugimi besedami je povlečena metrika $x^{*}\tilde{g}$ konformno ekvivalentna metriki $g$, kar pomeni, da za pozitivno funkcijo $\mu > 0$ na $M$ velja $x^{*}\tilde{g} = \mu g$.

Naj bo ploskev $M$ orientabilna in $x \colon M \to \R^{n}$ imerzija razreda $\mathcal{C}^2$. Potem preslikava $x$ določa enolično strukturo Riemannove ploskve na $M$, kjer je $x$ konformna imerzija. Zato bomo v nadaljevanju obravnavali Riemannove ploskve in pripadajoče konformne imerzije v Evklidski prostor.
Prvi rezultat, ki ga navajamo, opisuje ekvivalentne pogoje minimalnosti ploskve $M$.

\begin{izrek}
Naj bo $M$ odprta Riemannova ploskev, $n \geq 3$ in $x = (x_1, \dots , x_n) \colon M \to \R^{n}$ konformna imerzija razreda $\mathcal{C}^2$. Naslednje trditve so ekvivalentne:
\begin{enumerate}
	\item $x$ je minimalna ploskev.
	\item Vektorsko polje povprečne ukrivljenosti preslikave $x$ je ničelno, tj.~$\textbf{\textup{H}} = 0$.
	\item $x$ je harmonična, tj.~$\Delta{x} = 0$.
	\item 1-forma $ \partial{x} = (\partial{x_1}, \dots , \partial{x_n})$ z vrednostmi v $\C^{n}$ je holomorfna in velja
			\begin{equation}
			(\partial{x_1})^2 + \cdots + (\partial{x_n})^2 = 0.
			\end{equation}
	\item Naj bo $\theta$ holomorfna 1-forma na $M$, ki ni nikjer enaka $0$. Potem je preslikava $f = 2\partial{x} / \theta \colon M \to \C^{n}$ holomorfna z 				vrednostmi v \emph{ničelni kvadriki}
			\begin{equation} \label{ničelna-kvadrika}		
			\textbf{\textup{A}} = \{ (z_1, \dots , z_n) \in \C^{n} ; \ z_{1}^{2} + \cdots + z_{n}^{2} = 0 \}.
			\end{equation}	
		Nadalje je Riemannova metrika na $M$, inducirana s konformno imerzijo $x$, enaka
			\begin{align}
			g &= x^{*} ds^2 = |dx_1|^2 + \cdots + |dx_n|^2 = 2 (|\partial{x_1}|^2 + \cdots |\partial{x_n}|^2).
			\end{align}			
\end{enumerate}
\end{izrek}

\begin{definicija}
Naj bo $x \colon M \to \R^{n}$ harmonična preslikava. Njen \emph{pretok} je homomorfizem grup $\textup{Flux}_{x} \colon H_{1} (M, \Z) \to \R^{n}$, definiran s predpisom 
\begin{equation}
\textup{Flux}_{x} ([C]) = \int_{C} {d^{c} x}.
\end{equation}
\end{definicija}

V definiciji pretoka je $[C] \in H_{1} (M, \Z),$ integral pa je odvisen le od homološkega razreda poti $C$, zato bomo v nadaljevanju pisali kar $\textup{Flux}_{x} (C)$.

\begin{definicija}
\begin{enumerate}
\item
Naj bo $M$ odprta Riemannova ploskev in $n \geq 3$. Holomorfno imerzijo $z = (z_{1}, \dots , z_{n}) \colon M \to \C^{n}$, za katero velja
$(\partial{z_{1}})^2 + \cdots + (\partial{z_{n}})^2 = 0$, imenujemo \emph{holomorfna ničelna krivulja} v $\C^{n}$.
\item
Naj bo $z = x + \imath y \colon M \to \C^{n}$ holomorfna ničelna krivulja. Njena realni del in imaginarni del, $x, y \colon M \to \R^{n}$ imenujemo \emph{konjugirani minimalni ploskvi}.
\item
Naj bo $t \in \R$. Predstavnike 1-parametrične družine $x^{t} = \Re{(e^{\imath t} z)} \colon M \to \R^{n}$ imenujemo \emph{pridružene minimalne ploskve} holomorfne ničelne krivulje $z$.
\end{enumerate}
\end{definicija}

\begin{izrek}[Weierstrassova predstavitev konformnih minimalnih ploskev in holomorfnih ničelnih krivulj]
Naj bo $n \geq 3$ in $M$ odprta Riemannova ploskev, na kateri definiramo holomorfno 1-formo $\Phi = (\phi_{1}, \dots , \phi_{n})$ z vrednostmi v $\C^{n}$, ki je povsod neničelna, in zadošča 
\begin{enumerate}
\item $ \sum_{j=1}^{n} \phi_{j}^{2} = 0$,
\item $ \Re \int_{C} \Phi = 0 $ za vse $[C] \in H_{1} (M, \Z)$.
\end{enumerate}
Potem za poljuben izbor točk $p_0 \in M$ in $x_0 \in \R^{n}$ predpis $x \colon M \to \R^{n}$,
\begin{align} \label{eq:Wstrass-kmi}
x(p) = x_0 + \Re \int_{p_0}^{p} \Phi, \ p \in M,
\end{align}
podaja dobro definirano konformno minimalno imerzijo. Zanjo velja
\begin{align}
2 \partial{x} = \Phi \quad \text{in} \quad g = x^{*} ds^2 = |dx|^2 = \frac{1}{2} |\Phi|^2.
\end{align}
%
Če velja še
$ \int_{C} \Phi = 0 \ \text{za vse} \ [C] \in H_{1} (M, \Z) $,
potem za poljuben izbor točk $p_0 \in M$ in $z_0 \in \C^{n}$ predpis $z \colon M \to \C^{n}$,
\begin{align} \label{eq:Wstrass-hnk}
z(p) = z_0 + \int_{p_0}^{p} \Phi, \ p \in M,
\end{align}
podaja dobro definirano holomorfno ničelno krivuljo. Zanjo velja
\begin{align}
\partial{z} = \Phi \quad \text{in} \quad z^{*} ds^2 = |dz|^2 = |\partial{z}|^2 = |\Phi|^2.
\end{align}
\end{izrek}

\begin{opomba}
Vsaka konformna minimalna imerzija $x \colon M \to \R^{n}$ je oblike~\ref{eq:Wstrass-kmi} in vsaka holomorfna ničelna krivulja $z \colon M \to \C^{n}$ je oblike~\ref{eq:Wstrass-hnk}. Prav zato je Weierstrassova predstavitev elegantna metoda za konstrukcijo opisanih preslikav.
\end{opomba}

Če konformno minimalno imerzijo $x \colon M \to \R^{n}$ poznamo, potem pripadajočo povsod neničelno holomorfno 1-formo $\Phi = 2 \partial{x}$ z vrednostmi v $\C^{n}$ imenujemo \emph{Weierstrassovi podatki} preslikave $x$. 
Analogno, za holomorfno ničelno krivuljo $z \colon M \to \C^{n}$ pripadajočo 1-formo $\Phi = \partial{z} = dz$ imenujemo \emph{Weierstrassovi podatki} preslikave $z$.

\begin{definicija}
\emph{Jordanov lok} je pot v ravnini, ki je topološko izomorfna intervalu $[0,1]$.
\emph{Jordanova krivulja} je ravninska krivulja, ki je topološko ekvivalentna enotski krožnici.
\end{definicija}

\begin{definicija}
Naj bo $M$ gladka ploskev, $K$ končna unija paroma disjunktnih kompaktnih domen s kosoma zvezno odvedljivimi robovi v $M$ ter $E = S \setminus K^\circ$ unija končno mnogo paroma disjunktnih gladkih Jordanovih lokov in zaprtih Jordanovih krivulj, ki se dotikajo $K$ kvečjemu v svojih krajiščih in sekajo rob $K$ transverzalno. Kompaktno podmnožico v $M$ oblike $S = K \cup E$ imenujemo \emph{dopustna množica}.
\end{definicija}

\begin{definicija}
Naj bo $M$ povezana odprta Riemannova ploskev ali kompaktna Riemannova ploskev z robom, na kateri je definirana povsod neničelna holomorfna 1-forma $\Theta$. Konformno minimalno imerzijo $x \colon M \to \R^{n}$ imenujemo:
\begin{enumerate}
\item \emph{ravna}, če je slika $x(M)$ vsebovana v afini ravnini v $\R^{n}$; sicer pravimo, da je $x$ \emph{neravna};
\item \emph{polna}, če je preslikava $f = 2 \partial{x} / \Theta \colon M \to \textbf{\textup{A}}_{*}^{n-1}$ polna, tj.\ $\C$-linearna ogrinjača slike $f(M)$ je enaka $\C^{n}$;
\item \emph{neizrojena}, če slika $x(M)$ ni vsebovana v nobeni hiperravnini v $\R^{n}$. 
\end{enumerate}
\end{definicija}

V dimenziji  $n=3$ za konformno minimalno imerzijo vsi zgornji pojmi sovpadajo. V višjih dimenzijah ($n \geq 4$) veljata implikaciji 
\[ \text{polna} \ \Rightarrow \ \text{neizrojena} \ \Rightarrow \ \text{neravna}. \]

%
%Izreki o aproksimaciji in interpolaciji minimalnih ploskev
\section{Izreki o aproksimaciji in interpolaciji minimalnih ploskev}

Naj bosta $M$ in $X$ kompleksni mnogoterosti. Prostor holomorfnih presikav $M \to X$ označimo z $\mathcal{O}(M,X)$.
Če je $K$ kompaktna podmnožica v $M$, množico preslikav $K \to X$ razreda $\mathcal{C}^{r}(M)$, ki so holomorfne v notranjosti $K^\circ \subset K$, označimo z $\mathcal{A}^{r}(K,X)$.
V primeru, ko je $X = \C$, ustrezna prostora označimo z $\mathcal{O}(M)$ oziroma $\mathcal{A}^{r}(K)$.

Naj bo $M$ odprta Riemannova ploskev in $n \geq 3$. Prostor konformnih minimalnih imerzij $M \to \R^{n}$ označimo s $\textup{CMI}(M, \R^{n})$, prostor holomorfnih ničelnih imerzij $M \to \C^{n}$ pa z $\textup{NC}(M, \C^{n})$.
Nadalje $\textup{CMI}_{full}(M, \R^{n})$ in $\textup{CMI}_{nf}(M, \R^{n})$ označujeta prostora polnih oziroma neravnih konformnih minimalnih imerzij. Velja inkluzija $\textup{CMI}_{full}(M, \R^{n}) \subset \textup{CMI}_{nf}(M, \R^{n})$.
Podobno je $\textup{NC}_{full}(M, \C^{n}) \subset \textup{NC}_{nf}(M, \C^{n})$ v primeru polnih ter neravnih holomorfnih ničelnih krivulj.

Če je $M$ kompaktna omejena Riemannova ploskev z nepraznim gladkim robom $bM$ in $r \in \N$, tedaj prostor konformnih minimalnih imerzij $M \to \R^{n}$ razreda $\mathcal{C}^{r}(M)$ označimo s $\textup{CMI}^{r}(M, \R^{n})$, prostor holomorfnih ničenih imerzij $M \to \C^{n}$ razreda $\mathcal{A}^{r}(M)$ pa z $\textup{NC}^{r}(M, \C^{n})$.

\begin{lema} \label{lema:neravna f}
Naj bo $M$ povezana Riemannova ploskev in $\textbf{\textup{A}}_{*}$ punktirana ničelna kvadrika.
Holomorfna preslikava $f \colon M \to \textbf{\textup{A}}_{*}$ je neravna natanko tedaj, ko je linearna ogrinjača tangentnih prostorov 
$T_{f(p)} A \subset T_{f(p)} \C^{n}$ po vseh $p \in M$ enaka $\C^{n}$.
\end{lema}

\begin{dokaz}
Oglejmo si preslikavo 
$\Phi \colon \C^{n} \to \C$, definirano s predpisom $\Phi(z) = \sum_{j=1}^{n} z_{j}^{2}$.
Ničelno kvadriko~\ref{ničelna-kvadrika} tedaj lahko zapišemo v obliki $\textbf{\textup{A}} = \Phi^{-1}( \{0 \} )$.
Njen tangentni prostor v točki $z = (z_{1}, \dots , z_{n}) \in \C^{n}$ je enak jedru diferenciala, ki kvadriko določa, zato je
\[ T_{z} \textbf{\textup{A}} = \ker (d \Phi_{z}) = \ker (z \mapsto \sum_{j=1}^{n} z_{j} dz_{j}). \]

Naj bosta $z, \ w \in \C_{*}^{n}$. Potem sta njuna tangentna prostora enaka, $ T_{z} \textbf{\textup{A}} = T_{w} \textbf{\textup{A}} $, natanko tedaj, ko je $z_{j} = \lambda w_{j}$ za vse $j = 1, \dots , n$ in nek $\lambda \in \C$, kar je ekvivalentno pogoju, da sta vektorja $z$ in $w$ kolinearna.

Po definiciji je preslikava $f$ neravna, če njena slika $f(M)$ ni vsebovana v nobeni afini kompleksni premici v $\C^{n}$. Skupaj z zgornjim je slednje ekvivalnetno 
$ Lin \{T_{f(p)} \textbf{\textup{A}} ; \ p \in M \} = \C^{n}$, kar smo želeli dokazati.
\end{dokaz}

\begin{definicija}
Naj bo $S = K \cup E$ dopustna podmnožica Riemannove ploskve $M$ in $\Theta$ povsod neničelna holomorfna 1-forma, definirana v okolici $S \subset M$.
Naj bosta $n \geq 3$ in $r \in \N$. \emph{Posplošena konformna minimalna imerzija} $S \to \R^{n}$ razreda $\mathcal{C}^{r}$ je par $(x, f \Theta)$, kjer je $x \colon S \to \R^{n}$ preslikava razreda  $\mathcal{C}^{r}$, njena zožitev na $S^\circ = K^\circ$ je konformna minimalna imerzija in preslikava $f \in \mathcal{A}^{r-1}(S, \textbf{\textup{A}}_{*})$ zadošča naslednjima pogojema:
\begin{enumerate}
\item na množici $K$ velja $f \Theta = 2 \partial x$;
\item za vsako gladko pot $\alpha$ v $M$, ki parametrizira povezano komponento $E = \overline{S \setminus K}$ velja $ \Re(\alpha^{*}(f \Theta)) = \alpha^{*}(dx) = d(x \circ \alpha)$.
\end{enumerate}
%
Posplošena konformna minimalna imerzija $(x, f \Theta)$ je \emph{neravna} oziroma \emph{polna} natanko tedaj, ko je preslikava $f \in \mathcal{A}^{r-1}(S, \textbf{\textup{A}}_{*})$ neravna oziroma polna na vsaki relativno odprti podmnožici $S$.
\end{definicija}

Prostor posplošenih konformnih minimalnih imerzij $S \to \R^{n}$ razreda $\mathcal{C}^{r}$ označimo z $\textup{GCMI}^{r}(S, \R^{n})$. Analogno kot v primeru konformnih minimalnih imerzij velja 
\[ \textup{GCMI}_{full}^{r}(S, \R^{n}) \subset \textup{GCMI}_{nf}^{r}(S, \R^{n}) \subset \textup{GCMI}^{r}(S, \R^{n}). \]

\begin{opomba}
Diferencial $d$ v kompleksnem ima obliko $d = \partial + \bar{\partial}$. \emph{Konjugirani difernecial} $d^{c}$ je enak $d^{c} = i(\bar{\partial} - \partial) = 2 \Im(\partial)$. Zato velja $d + i d^{c} = 2 \partial$ oziroma drugače, $\Re(2 \partial) = dx$.
Prvi pogoj iz definicije posplošene konformne minimalne imerzije pravi $f \Theta = 2 \partial$, od koder sledi $\Re(f \Theta) = \Re(2 \partial) = dx$. Zato je drugi pogoj iz zgornje definicije skladen s prvim.
\end{opomba}

Tudi za posplošene konformne minimalne imerzije velja Weierstrassova formula.
Naj bo $S$ povezana dopustna množica in $(x, f \Theta) \in \textup{GCMI}^{r}(S, \R^{n})$. Za poljubno točko $p_{0} \in S$ in poznano preslikavo $f$ lahko preslikavo $x \colon S \to \R^{n}$ konstruiramo s formulo
\begin{align} \label{eq:Wstrass-gcmi}
x(p) = x(p_{0}) + \Re \int_{p_0}^{p} f \Theta, \ p \in S.
\end{align} 
Obratno, če za preslikavo $f \in \mathcal{A}^{r-1}(S, \textbf{\textup{A}}_{*})$ velja $ \Re \int_{C} f \Theta = 0$ za vsako sklenjeno krivuljo $C$ v $S$, potem $f$ določa posplošeno konformno minimalno imerzijo, dano z Weierstrassovo formulo~\ref{eq:Wstrass-gcmi}.

\begin{definicija}
Naj bo $S = K \cup E$ dopustna podmnožica Riemannove ploskve $M$ in $\Theta$ povsod neničelna holomorfna 1-forma, definirana v okolici $S \subset M$.
Naj bosta $n \geq 3$ in $r \in \N$. \emph{Posplošena ničelna krivulja} $S \to \C^{n}$ razreda $\mathcal{C}^{r}$ je par $(z, f \Theta)$, kjer preslikavi $z \in \mathcal{A}^{r}(S, \C^{n})$ in $f \in \mathcal{A}^{r-1}(S, \textbf{\textup{A}}_{*})$ zadoščata naslednjima pogojema:
\begin{enumerate}
\item na množici $K$ velja $f \Theta = dz = \partial z$;
\item za vsako gladko pot $\alpha$ v $M$, ki parametrizira povezano komponento $E = \overline{S \setminus K}$ velja $ \alpha^{*}(f \Theta)) = \alpha^{*}(dz) = d(z \circ \alpha)$.
\end{enumerate}
%
Posplošena ničelna krivulja $(z, f \Theta)$ je \emph{neravna} oziroma \emph{polna} natanko tedaj, ko je preslikava $f \in \mathcal{A}^{r-1}(S, \textbf{\textup{A}}_{*})$ neravna oziroma polna na vsaki relativno odprti podmnožici $S$.
\end{definicija}

Prostori neravnih, polnih in posplošenih ničelnih krivulj ustrezajo verigi inkluzij
\[ \textup{GNC}_{full}^{r}(S, \C^{n}) \subset \textup{GNC}_{nf}^{r}(S, \C^{n}) \subset \textup{GNC}^{r}(S, \C^{n}). \]
Za povezano dopustno množico $S$, $(z, f \Theta) \in \textup{GNC}^{r}(S, \C^{n})$, znano preslikavo $f$ in točko $p_{0} \in S$ preslikavo $z \colon S \to \C^{n}$ konstruiramo s pomočjo Weierstrassove formule
\begin{align} \label{eq:Wstrass-gnc}
z(p) = z(p_{0}) + \int_{p_0}^{p} f \Theta, \ p \in S.
\end{align} 
Velja tudi obrat; preslikava $f \in \mathcal{A}^{r-1}(S, \textbf{\textup{A}}_{*})$, ki zadošča $\int_{C} f \Theta = 0$ za vsako sklenjeno krivuljo $C$ v $S$, določa posplošeno ničelno krivuljo, dano z Weierstrassovo formulo~\ref{eq:Wstrass-gnc}.

\begin{definicija}
Naj bo $M$ povezana odprta Riemannova ploskev. Naj bo $\Theta$ fiksna povsod neničelna holomorfna 1-forma na $M$. Naj bo $\mathcal{C} = \{C_1, \dots , C_{l} \}$ družina gladkih orientiranih vloženih lokov in zaprtih Jordanovih krivulj v $M$ ter $C = \cup_{i=1}^{l} C_{i}$.
Družini $\mathcal{C}$ in številu $n \in \N$ priredimo \emph{periodno preslikavo}
\begin{align}
\mathcal{P} = (\mathcal{P}_1, \dots , \mathcal{P}_{l}) \colon \mathcal{C}(C, \C^{n}) \to (\C^{n})^{l}, \nonumber \\
\mathcal{P}_{i}(f) = \int_{C_{i}} f \Theta, \ i=1, \dots , l.
\end{align}
Tu je $f \in \mathcal{C}(C, \C^{n})$ in $\mathcal{P}_{i}(f) \in \C^{n}$.
\end{definicija}

\begin{opomba}
Znano je, da vsaka odprta Riemannova ploskev $M$ premore lokalno biholomorfno preslikavo $M \to \C^{n}$, torej povsod neničelno eksaktno holomorfno 1-formo. Zato je predpostavka o izboru 1-forme v zgornji definiciji smiselna.
\end{opomba}

\begin{lema} \label{lema:P-D-sprej}
Naj bo $M$ odprta Riemannova ploskev in $S = K \cup E$ njena dopustna podmnožica. Naj bo $\mathcal{C} = \{C_1, \dots , C_{l} \}$ taka družina gladkih orientiranih Jordanovih krivulj in lokov v $S$, da je unija $C = \cup_{i=1}^{l} C_{i}$ Rungejeva v $S$.
Naj za neko število $r \in \Z_{+}$ preslikava $f$ pripada razredu $\mathcal{A}^{r}(S, \textbf{\textup{A}}_{*})$.
Nadalje predpostavimo, da vsaka krivulja $C_{i} \in \mathcal{C}$ vsebuje netrivialen lok $I_{i} \in C_{i}$, disjunkten z $\cup_{i \neq j}C_{j}$, preslikava $f \colon I_{i} \to \textbf{\textup{A}}_{*}$ pa je neravna.

Potem obstaja odprta okolica $U \subset \C^{ln}$ točke $0$ in preslikava $\Phi_{t} \in \mathcal{A}^{r} (S \times U, \textbf{\textup{A}}_{*})$, tako da velja
$\Phi_{t}(\cdot, 0) = f$ in je preslikava
\begin{align} \label{PD-lastnost}
	 \frac{\partial}{\partial t} \Big|_{t=0} \mathcal{P}(\Theta_{f}(\cdot, t)) \colon (\C^{n})^{l} \to (\C^{n})^{l} \ \text{izomorfizem.}
\end{align}

Nadalje, za končno podmnožico $P \subset S$ lahko preslikavo $\Theta_{f}$ izberemo tako, da se za $t \in U$ preslikave $\Theta_{f}(\cdot, t) \colon S \to\textbf{\textup{A}}_{*}$ ujemajo z $f$ v vsaki točki $P \setminus S^\circ$, v točkah $P \cap S^\circ$ pa se z $f$ ujemajo do danega končnega reda.

Za vsako preslikavo $f_{0} \in \mathcal{A}^{r}(S, \textbf{\textup{A}}_{*})$, ki zadošča zgornjim predpostavkam, obstaja okolica $\Omega \subset \mathcal{A}^{r}(S, \textbf{\textup{A}}_{*})$ in holomorfna preslikava $f \mapsto \Theta_{f}$, $f \in \Omega$ z zgornjimi lastnostmi.
\end{lema}

\begin{definicija}
Preslikavo $\Theta_{f}$, ki ustreza Lemi~\ref{lema:P-D-sprej} imenujemo \emph{periodno dominantni sprej} preslikav $S \to \textbf{\textup{A}}_{*}$ za družino krivulj $\mathcal{C}$ z \emph{jedrom} $\Theta_{f}(\cdot, 0) = f$. Lastnosti~\ref{PD-lastnost} pravimo \emph{periodno dominantna lastnost}.
\end{definicija}

\begin{dokaz}
Prvi del leme bomo dokazali tako, da bomo konstruirali periodno dominantni sprej, ki zadošča periodno dominantni lastnosti. Potrebovali bomo Lemo~\ref{lema:neravna f}, Bishop-Mergelyanov izrek o aproksimaciji~\ref{izr:Bishop-Mergelyan} in pojem toka vektorskega polja.
Zaradi enostavnosti postavimo $r=0$ (za $r>0$ dokaz poteka analogno).

Po predpostavki je za vse $i \in \{ 1, \dots, n \}$ lok $I_{i} \subset C_{i} \in \mathcal{C}$ netrivialen, za katerega velja $I_{i} \cap \cup_{i \neq j} C_{j} = \emptyset$ in je zožitev preslikave $f|_{I_{i}}$ neravna. Po Lemi~\ref{lema:neravna f} obstajajo točke $p_{i,j} \in I_{i}$ in holomorfna vektorska polja $V_{i,j}$ na $\C^{n}$, $j \in \{1, \dots, n \}$, ki so tangentna na $\textbf{\textup{A}}$, tako da je $Lin \{ V_{i,j}(f(p_{i,j})) ; \ j = 1, \dots, n \} = \C^{n}$ za vse $i$.

Za $k = 1, \dots l$ označimo $t_{k} = (t_{k,1}, \dots, t_{k,n}) \in \C^{n}$ in $t = (t_{1}, \dots, t_{n}) \in \C^{nl}$. Naj $\Phi_{t}^{i,j}$ označuje tok vektorskega polja $V_{i,j}$.
Izberimo tako odprto okolico $U_{0} \subset \C^{nl}$ točke $0$, da za vse $t \in U_{0}$ in $p \in S$ predpis
\begin{align} \label{predpis-tokovi}
(p, t) \mapsto \Phi_{t_{1,1}}^{1,1} \circ \cdots \circ \Phi_{t_{1,n}}^{1,n} \circ \Phi_{t_{2,1}}^{2,1} \circ \cdots \circ \Phi_{t_{l,n}}^{l,n} (f(p))
\end{align}
podaja dobro definirano preslikavo $S \times U_{0} \to \textbf{\textup{A}}_{*}$.
Sedaj za vse pare $(i,j)$ izberimo gladke preslikave $g_{i,j} \colon C \to \C$, pri čemer je nosilec $g_{i,j}$ vsebovan v majhnem delu loka $I_{i}$ okrog točke $p_{i,j} \in I_{i}$.
Modificirana preslikava~\ref{predpis-tokovi}, $\Phi \colon C \times U_{1} \to \textbf{\textup{A}}_{*}$,
\begin{align} \label{predpis-Phi}
\Phi(p,t) = \Phi_{g_{1,1}(p)t_{1,1}}^{1,1} \circ \cdots \circ \Phi_{g_{l,n}(p)t_{l,n}}^{l,n} (f(p)),
\end{align}
kjer je $U_{1} \subset \C^{nl}$ primerno majhna odprta okolica točke $0$, je tedaj dobro definirana, za vse $p \in C$ pa je preslikava $\Phi(p, \cdot) \colon U_{1} \to \textbf{\textup{A}}_{*}$ holomorfna. Po lastnostih toka vektorskega polja sledi še $\Phi(p,0) = f(p)$ in
\begin{align} \label{dPhi/dt}
\frac{\partial \Phi(p,t)}{\partial t_{m,j}} \Big|_{t=0} = g_{m,j}(p) \cdot V_{m,j}(f(p)).
\end{align}

Naj bo $\mathcal{P} = (\mathcal{P}_{1}, \dots, \mathcal{P}_{n})$ periodna preslikava, prirejena družini krivulj $\mathcal{C}$. Z uporabo enakosti~\ref{dPhi/dt} dobimo za vse indekse $i, m \in \{1, \dots, l \}$ in $j \in \{1, \dots, n \}$
\begin{align}
\frac{\partial \mathcal{P}_{i}(\Phi(\cdot, t))}{\partial t_{m,j}} \Big|_{t=0} = \frac{\partial}{\partial t_{m,j}} \Big|_{t=0} \int_{C_{i}} \Phi(\cdot, t) \cdot \Theta = \int_{C_{i}} g_{m,j} \cdot (V_{m,j} \circ f) \cdot \Theta \in \C^{n}.
\end{align}
Matrika diferencialov~\ref{PD-lastnost} iz leme je sestavljena iz blokov velikosti $n \times n$, ki pripadajo indeksom $i, m \in \{1, \dots, l \}$. Z ustrezno izbiro preslikav $g_{i,j}$ opisanih zgoraj lahko dosežemo, da je matrika bločno diagonalna z obrnljivimi bloki na diagonali. S tem postane celotna matrika obrnljiva.

V naslednjem koraku bomo modificirali še preslikavo $\Phi$, kar nam bo dalo iskani periodno dominantni prej.
Preslikave $g_{i,j}$ so definirane na množici $C$, ki je po predpostavki Rungejeva v $S$. Bishop-Mergelyanov izrek o aproksimaciji pove, da vsako funkcijo $g_{i,j}$ lahko enakomerno na $C$ aproksimiramo s holomorfnimi funkcijami $\tilde{g}_{i,j}$ v okolici $S$.

Definirajmo preslikavo $\Phi_{f} \colon S \times U \to \textbf{\textup{A}}_{*}$ tako, da v predpisu~\ref{predpis-Phi} nadomestimo $g_{i,j}$ z novimi funkcijami $\tilde{g}_{i,j}$ in je $U \subset U_{1} \subset \C^{nl}$ odprta okolica izhodišča. Po konstrukciji takšna preslikava $\Phi_{f}$ zadošča sklepom leme, zato je periodno dominantni sprej, ki smo ga iskali.
\end{dokaz}

\begin{trditev}
Naj bo $M$ odprta Riemannova ploskev, $\Theta$ povsod neničelna holomorfna 1-forma na $M$, $S$ povezana dopustna množica, ki je Rungejeva v $M$ in $A=\{a_{1}, \dots , a_{k} \} \subset S$. Naj bosta $r, s \in \N$. 
Potem lahko vsako posplošeno konformno minimalno imerzijo $(x, f\Theta) \in GCMI^{r}(S,\R^{n})$ aproksimiramo s konformnimi minimalnimi imerzijami $X \colon M \to \R^{n}$ razreda $\mathcal{C}^{r}$, za katere velja $\textup{Flux}_{X} = \textup{Flux}_{x}$. 
\end{trditev}

\begin{izrek}
Naj bo $M$ odprta Riemannova ploskev, $\Theta$ povsod neničelna holomorfna 1-forma na $M$, $n \geq 3$ in $r \geq 1$.
Naj bo $S$ dopustna Rungejeva množicca v $M$ in $\Lambda$ zaprta diskretna podmnožica $M$. 
Naj bo $x \colon S \to \R^{n}$ posplošena konformna minimalna imerzija razreda $\mathcal{C}^{r}(S, \R^{n})$, ki je konformna minimalna imerzija v okolici vsake točke iz $\Lambda$.

Za izbrane $\varepsilon > 0$, preslikavo $k \colon \Lambda \to \N$ in homomorfizem grup $\mathfrak{p} \colon H_{1}(M,\Z) \to \R^{n}$, $\mathfrak{p}|_{H_{1}(S,\Z)} = \textup{Flux}_{x}$, obstaja konformna minimalna imerzija $\tilde{x} \colon M \to \R^{n}$, za katero velja:
\begin{enumerate}
\item $||\tilde{x} - x||_{\mathcal{C}^{r}(S)} < \varepsilon$;
\item Razlika $\tilde{x}-x$ je ničelna do reda $k(p)$ v vsaki točki $p\in \Lambda$;
\item $Flux_{\tilde{x}} = \mathfrak{p}$ na $H_{1}(M,\Z)$;
\item Če je $n\geq5$ in je $x \colon \Lambda \to \R^{n}$ injektivna preslikava, potem je $\tilde{x}$ injektivna imerzija;
\item Če je $n=4$ in ima $x$ enostavne dvojne točke na množici $\Lambda$, potem je $\tilde{x}$ imerzija z enostavnimi dvojnimi točkami na $\Lambda$.
\end{enumerate}
\end{izrek}

% Literatura:
% Primer navajanja na http://www.fmf.uni-lj.si/storage/24240/LiteraturaM.pdf,
% ampak bi moral stil poskrbeti za vse. Reference se uredijo po abecedi.
% Če nobena izbira izmed @book, @atricle,... ni ok, potem se lahko vse napiše v
% @misc pod note={} in deluje tako kot normalen LaTeX.
% Komentar v bib datoteki se naredi samo s parom { }
% Za urejanje literature avtor priporoča program Jabref, ki zna tudi avtomatsko
% okrajšati imena revij. Za pravilno sortiranje vnosov brez avtorja, uporabite
% polje key={ }, kot v primeru.
% V primeru napak ustvarite issue na GitHubu ali pišite na jure.slak@fmf.uni-lj.si.
\cleardoublepage                           % na desni strani
\phantomsection                            % da prav delujejo hiperlinki
\addcontentsline{toc}{section}{\bibname}   % dodajmo v kazalo
\bibliographystyle{fmf-sl}                 % uporabljen stil je v datoteki fmf-sl.bst, na voljo tudi angleška verzija
\bibliography{\literatura}                 % literatura je v datoteki, definirani na začetku
% TeXStudio zmede \ zgoraj, tako da lahko notri napišeš dejansko ime .bib datoteke, če ti
% ne delajo predlogi citatov.

% Za stvarno kazalo
\cleardoublepage                           % na desni strani
\phantomsection                            % da prav delujejo hiperlinki
\addcontentsline{toc}{section}{\indexname} % dodajmo v kazalo
\printindex

\end{document}
