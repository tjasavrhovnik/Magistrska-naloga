% !TeX spellcheck = sl_SI
% vim: set spell spelllang=sl:
% za preverjanje črkovanja, če se uporablja Texstudio ali vim
\documentclass[12pt,a4paper,twoside]{article}
\usepackage[utf8]{inputenc}  % pravilno razpoznavanje unicode znakov

% NASLEDNJE UKAZE USTREZNO POPRAVI
\newcommand{\program}{Matematika} % ime studijskega programa
\newcommand{\imeavtorja}{Tjaša Vrhovnik} % ime avtorja
\newcommand{\imementorja}{prof.~dr.~Franc Forstnerič} % akademski naziv in ime mentorja, uporabi poln naziv, prof.~dr.~, doc.~dr., ali izr.~prof.~dr.
\newcommand{\imesomentorja}{} % akademski naziv in ime somentorja, če ga imate
\newcommand{\naslovdela}{Minimalne ploskve}
\newcommand{\letnica}{2021} % letnica magistriranja
\newcommand{\opis}{}  % Opis dela v eni povedi. Ne sme vsebovati matematičnih simbolov v $ $.
\newcommand{\kljucnebesede}{} % ključne besede, ločene z \sep, da se PDF metapodatki prav procesirajo
\newcommand{\keywords}{} % ključne besede v angleščini
\newcommand{\organization}{Univerza v Ljubljani, Fakulteta za matematiko in fiziko} % fakulteta
\newcommand{\literatura}{literatura}  % pot do datoteke z literaturo (brez .bib končnice)
\newcommand{\sep}{, }  % separator med ključnimi besedami v besedilu
% KONEC PODATKOV

\usepackage{bibentry}         % za navajanje literature v programu dela s celim imenom
\nobibliography{\literatura}
\newcommand{\plancite}[1]{\item[\cite{#1}] \bibentry{#1}} % citiranje v programu dela

\usepackage{filecontents}  % za pisanje datoteke s PDF metapodatki
\usepackage{silence} \WarningFilter{latex}{Overwriting file}  % odstrani annoying warning o obstoju datoteke
% datoteka s PDF metapodatki, zgenerira se kot magisterij.xmpdata
\begin{filecontents*}{\jobname.xmpdata}
  \Title{\naslovdela}
  \Author{\imeavtorja}
  \Keywords{\kljucnebesede}
  \Subject{matematika}
  \Org{\organization}
\end{filecontents*}

\usepackage[a-1b]{pdfx}  % zgenerira PDF v tem PDF/A-1b formatu, kot zahteva knjižnica
\hypersetup{bookmarksopen, bookmarksdepth=3, colorlinks=true,
  linkcolor=black, anchorcolor=black, citecolor=black, filecolor=black,
  menucolor=black, runcolor=black, urlcolor=black, pdfencoding=auto,
  breaklinks=true, psdextra}

\usepackage[slovene]{babel}  % slovenščina
\usepackage[T1]{fontenc}     % naprednejše kodiranje fonta
\usepackage{amsmath,amssymb,amsfonts,amsthm} % matematični paketi
\usepackage{graphicx}     % za slike
\usepackage{emptypage}    % prazne strani so neoštevilčene, ampak so štete
\usepackage{units}        % fizikalne enote kot \unit[12]{kg} s polovico nedeljivega presledka, glej primer v kodi
\usepackage{makeidx}      % za stvarno kazalo, lahko zakomentiraš, če ne rabiš
\makeindex                % za stvarno kazalo, lahko zakomentiraš, če ne rabiš
% oblika strani
\usepackage[
  top=3cm,
  bottom=3cm,
  inner=3.5cm,      % margini za dvostransko tiskanje
  outer=2.5cm,
  footskip=40pt     % pozicija številke strani
]{geometry}

% VEČ ZANIMIVIH PAKETOV
% \usepackage{array}      % več možnosti za tabele
% \usepackage[list=true,listformat=simple]{subcaption}  % več kot ena slika na figure, omogoči slika 1a, slika 1b
% \usepackage[all]{xy}    % diagrami
% \usepackage{doi}        % za clickable DOI entrye v bibliografiji
% \usepackage{enumerate}     % več možnosti za sezname

% Za barvanje source kode
% \usepackage{minted}
% \renewcommand\listingscaption{Program}

% Za pisanje psevdokode
% \usepackage{algpseudocode}  % za psevdokodo
% \usepackage{algorithm}
% \floatname{algorithm}{Algoritem}
% \renewcommand{\listalgorithmname}{Kazalo algoritmov}

% DRUGI TVOJI PAKETI:
% tukaj

\setlength{\overfullrule}{50pt} % označi predlogo vrstico
\pagestyle{plain}               % samo številka strani na dnu, nobene glave / noge

% ukazi za matematična okolja
\theoremstyle{definition} % tekst napisan pokončno
\newtheorem{definicija}{Definicija}[section]
\newtheorem{primer}[definicija]{Primer}
\newtheorem{opomba}[definicija]{Opomba}
\newtheorem{aksiom}{Aksiom}

\theoremstyle{plain} % tekst napisan poševno
\newtheorem{lema}[definicija]{Lema}
\newtheorem{izrek}[definicija]{Izrek}
\newtheorem{trditev}[definicija]{Trditev}
\newtheorem{posledica}[definicija]{Posledica}

\numberwithin{equation}{section}  % števec za enačbe zgleda kot (2.7) in se resetira v vsakem poglavju

% Matematični ukazi
\newcommand{\R}{\mathbb R}
\newcommand{\N}{\mathbb N}
\newcommand{\Z}{\mathbb Z}
\renewcommand{\C}{\mathbb C}
\newcommand{\Q}{\mathbb Q}

% \DeclareMathOperator{\tr}{tr}  % morda potrebuješ operator za sled ali kaj drugega?

% bold matematika znotraj \textbf{ }, tudi v naslovih, kot \omega spodaj
\makeatletter \g@addto@macro\bfseries{\boldmath} \makeatother

% Poimenuj kazalo slik kot ``Kazalo slik'' in ne ``Slike''
\addto\captionsslovene{
  \renewcommand{\listfigurename}{Kazalo slik}%
}

% če želiš, da se poglavja začnejo na lihih straneh zgoraj
% \let\oldsection\section
% \def\section{\cleardoublepage\oldsection}

%%%%%%%%%%%%%%%%%%%%%%%%%%%%%%%%%%%%%%%%%%
%%%%%%           DOCUMENT           %%%%%%
%%%%%%%%%%%%%%%%%%%%%%%%%%%%%%%%%%%%%%%%%%

\begin{document}

\pagenumbering{roman} % začnemo z rimskimi številkami
\thispagestyle{empty} % ampak na prvi strani ni številke

\noindent{\large
UNIVERZA V LJUBLJANI\\[1mm]
FAKULTETA ZA MATEMATIKO IN FIZIKO\\[5mm]
\program\ -- 2.~stopnja}
% ustrezno dopolni za IŠRM
\vfill

\begin{center}
  \large
  \imeavtorja\\[3mm]
  \Large
  \textbf{\MakeUppercase{\naslovdela}}\\[10mm]
  \large
  Magistrsko delo \\[1cm]
  Mentor: \imementorja \\[2mm] % ustrezno popravi spol
%   Somentor: \imesomentorja   % dodaj, če potrebno
\end{center}
\vfill

\noindent{\large Ljubljana, \letnica}

\cleardoublepage

%% sem pride IZJAVA O AVTORSTVU  -- SE NATISNE V VIS

% zahvala
\pdfbookmark[1]{Zahvala}{zahvala} %
\section*{Zahvala}
% end zahvala -- izbriši vse med zahvala in end zahvala, če je ne rabiš

\cleardoublepage

\pdfbookmark[1]{\contentsname}{kazalo-vsebine}
\tableofcontents

% list of figures
% \cleardoublepage
% \pdfbookmark[1]{\listfigurename}{kazalo-slik}
% \listoffigures
% end list of figures

\cleardoublepage

\section*{Program dela}
\addcontentsline{toc}{section}{Program dela} % dodajmo v kazalo

\section*{Osnovna literatura}
Literatura mora biti tukaj posebej samostojno navedena (po pomembnosti) in ne
le citirana. V tem razdelku literature ne oštevilčimo po svoje, ampak uporabljamo
okolje itemize in ukaz plancite, saj je celotna literatura oštevilčena na koncu.
\begin{itemize}
  \plancite{lebedev2009introduction}
  \plancite{gurtin1982introduction}
  \plancite{zienkiewicz2000finite}
  \plancite{STtemplate}
\end{itemize}

\vspace{2cm}
\hspace*{\fill} Podpis mentorja: \phantom{prostor za podpis}

% \vspace{2cm}
% \hspace*{\fill} Podpis somentorja: \phantom{prostor za podpis}

\cleardoublepage
\pdfbookmark[1]{Povzetek}{abstract}

\begin{center}
\textbf{\naslovdela} \\[3mm]
\textsc{Povzetek} \\[2mm]
\end{center}
Tukaj napišemo povzetek vsebine. Sem sodi razlaga vsebine in ne opis tega, kako je delo
organizirano.

\vfill
\begin{center}
\textbf{English translation of the title} \\[3mm] % prevod slovenskega naslova dela
\textsc{Abstract}\\[2mm]
\end{center}

An abstract of the work is written here. This includes a short description of
the content and not the structure of your work.

\vfill\noindent
\textbf{Math.~Subj.~Class.~(2010):} oznake kot 74B05, 65N99, na voljo so na naslovu
\url{http://www.ams.org/msc/msc2010.html} \\[1mm]
\textbf{Ključne besede:} \kljucnebesede \\[1mm]
\textbf{Keywords:} \keywords

\cleardoublepage

\setcounter{page}{1}    % od sedaj naprej začni zopet z 1
\pagenumbering{arabic}  % in z arabskimi številkami

\section{Uvod}

\section{Osnovni pojmi}

\section{Izerki o aproksimaciji in interpolaciji minimalnih ploskev}


% Literatura:
% Primer navajanja na http://www.fmf.uni-lj.si/storage/24240/LiteraturaM.pdf,
% ampak bi moral stil poskrbeti za vse. Reference se uredijo po abecedi.
% Če nobena izbira izmed @book, @atricle,... ni ok, potem se lahko vse napiše v
% @misc pod note={} in deluje tako kot normalen LaTeX.
% Komentar v bib datoteki se naredi samo s parom { }
% Za urejanje literature avtor priporoča program Jabref, ki zna tudi avtomatsko
% okrajšati imena revij. Za pravilno sortiranje vnosov brez avtorja, uporabite
% polje key={ }, kot v primeru.
% V primeru napak ustvarite issue na GitHubu ali pišite na jure.slak@fmf.uni-lj.si.
\cleardoublepage                           % na desni strani
\phantomsection                            % da prav delujejo hiperlinki
\addcontentsline{toc}{section}{\bibname}   % dodajmo v kazalo
\bibliographystyle{fmf-sl}                 % uporabljen stil je v datoteki fmf-sl.bst, na voljo tudi angleška verzija
\bibliography{\literatura}                 % literatura je v datoteki, definirani na začetku
% TeXStudio zmede \ zgoraj, tako da lahko notri napišeš dejansko ime .bib datoteke, če ti
% ne delajo predlogi citatov.

% Za stvarno kazalo
\cleardoublepage                           % na desni strani
\phantomsection                            % da prav delujejo hiperlinki
\addcontentsline{toc}{section}{\indexname} % dodajmo v kazalo
\printindex

\end{document}
